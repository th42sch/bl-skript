% !TeX spellcheck = de_DE_frami
\documentclass[fontsize=11pt, twoside=false, numbers=autoenddot]{scrbook}
\usepackage{bl_tafelanschriebe}

\pagestyle{plain}
% \pagestyle{scrheadings}
% % \chead{\headmark}
% \chead{\partmark}
% \renewcommand{\sectionmark}[1]{\markright{\textsl{#1}}}
% \renewcommand{\chaptermark}[1]{\markright{\textsl{#1}}{}}
% \renewcommand{\partmark}[1]{\markright{\textsl{#1}}}
\parindent0pt
\parskip\smallskipamount

\title{Tafelmitschriften zur Vorlesung \glqq Beschreibungslogik\grqq\\ im Sommersemester 2019}
\author{%
  Prof.\ Dr.\ Thomas Schneider\\[1pt]
  AG Theorie der Künstlichen Intelligenz \\[1pt]
  Fachbereich 3 \\
  \includegraphics[width=.4\linewidth]{logo_ub.jpg} \\[\baselineskip]~%
}
\date{Stand: \today}
\publishers{{\large Dieses Dokument ist noch unvollständig und wird regelmäßig aktualisiert.}}

\begin{document}

\maketitle
\tableofcontents

\refstepcounter{part}
% ===================================================================
% ===================================================================
% ===================================================================
\part{Grundlagen}

% ===================================================================
\section*{T2.1~ Beispiele für {\boldmath \ALC}-Konzepte}

Mit den Konzeptnamen
%
\parI
\begin{quote}
  \term{Student}, \term{Naturwissenschaft}, \term{Professor}, \term{Emeritus},
  \term{PflichtVL}, \term{VL}, \term{Einfach}, \mbox{\term{Interessant}},\\
  $A,B$
\end{quote}
%
\parI
und den Rollennamen
%
\parI
\begin{quote}
  \term{studiert}, \term{hält}, \term{hatÜbungsaufgabe},\\
  $r$
\end{quote} 
%
\parI
kann man z.\,B.\ folgende zusammengesetzte \ALC-Konzepte bilden:
%
\parI
\begin{itemize}
  \item
    $\term{Student} \sqcap \exists\term{studiert}.\term{Naturwissenschaft}$ \\
    (beschreibt Studierende, die mindestens eine Naturwissenschaft studieren)
    \parI
  \item
    $\term{Professor} \sqcap \term{Emeritus} \sqcap \forall\term{hält}.\lnot\term{PflichtVL}$ \\
    (beschreibt Professor*innen im Ruhestand, die keine Pflichtvorlesungen halten)
    \parI
  \item 
    $\term{VL} \sqcap \lnot\term{PflichtVL} \sqcap \forall\term{hatÜbungsaufgabe}.(\term{Einfach} \sqcup \term{Interessant})$ \\
    (beschreibt Wahlvorlesungen, bei denen alle Übungsaufgaben einfach oder interessant sind)
    \parI
  \item
    $A \sqcap \exists r.(\lnot B \sqcup \forall r.A)$
\end{itemize}
%
\parI
(Die Beschreibungen in Klammern werden eigentlich erst richtig klar, wenn die Semantik definiert ist.)

% ===================================================================
\section*{T2.2~ Beispiele für Interpretationen und Extensionen}

Wir betrachten die Interpretation $\Imc = (\Delta^\Imc,\cdot^\Imc)$ mit
%
\begin{align*}
  \Delta^\Imc            & = \{s_1,s_2,s_3,v_1,v_2\} \\[4pt]
  \term{Mensch}^\Imc     & = \{s_1,s_2,s_3\}         \\
  \term{Student}^\Imc    & = \{s_1,s_2,s_3\}         \\
  \term{Vorlesung}^\Imc  & = \{v_1,v_2\}             \\
  \term{PflichtVL}^\Imc  & = \{v_1\}                 \\
  \term{WahlVL}^\Imc     & = \{v_2\}                 \\
  \term{hört}^\Imc       & = \{(s_1,v_1),(s_2,v_1),(s_2,v_2),(s_3,v_1)\} \\
  \term{bekanntMit}^\Imc & = \{(s_1,s_2),(s_2,s_1),(s_1,s_1),(s_2,s_2),(s_3,s_3)\}.
\end{align*}
%
Jede Interpretation kann in offensichtlicher Weise als (knoten- und kantenbeschrifteter)
gerichteter Graph aufgefasst werden; für unsere Beispielinterpretation \Imc:
%
\begin{center}
  \begin{tikzpicture}[%
    >=Latex,
    every state/.style={draw=black,thin,fill=black!10,inner sep=1mm,minimum size=8mm},
    every edge/.style={draw=black,thin}
  ]
    \node[state] (s1)                                   {$s_1$};
    \node[state] (s2) [right=35mm of s1]                {$s_2$};
    \node[state] (s3) [right=35mm of s2]                {$s_3$};
    \node[state] (v1) [below right=25mm and 17mm of s1] {$v_1$};
    \node[state] (v2) [right=35mm of v1]                {$v_2$};
    
    \node [above=-1mm of s1] {\begin{tabular}{@{}c@{}}\term{Student}\\\term{Mensch}\end{tabular}};
    \node [above=-1mm of s2] {\begin{tabular}{@{}c@{}}\term{Student}\\\term{Mensch}\end{tabular}};
    \node [above=-1mm of s3] {\begin{tabular}{@{}c@{}}\term{Student}\\\term{Mensch}\end{tabular}};
    \node [below=0mm of v1] {\begin{tabular}{@{}c@{}}\term{Vorlesung}\\\term{PflichtVL}\end{tabular}};
    \node [below=0mm of v2] {\begin{tabular}{@{}c@{}}\term{Vorlesung}\\\term{WahlVL}\end{tabular}};
    
    \path[->] (s1) edge[loop left]     node[left]             {\term{bekanntMit}} ()
              (s1) edge[bend left =10] node[above]            {\term{bekanntMit}} (s2)
              (s2) edge[bend left =10] node[below]            {\term{bekanntMit}} (s1)
              (s2) edge[loop right]    node[right]            {\term{bekanntMit}} ()
              (s3) edge[loop right]    node[right]            {\term{bekanntMit}} ()
              (s1) edge                node[left =1mm]        {\term{hört}}       (v1)
              (s2) edge                node[right=1mm]        {\term{hört}}       (v1)
              (s2) edge                node[right=1mm,pos=.7] {\term{hört}}       (v2)
              (s3) edge                node[right=3mm,pos=.3] {\term{hört}}       (v1)
    ;
  \end{tikzpicture}
\end{center}
%
Beispiele für die Extensionen einiger zusammengesetzter Konzepte in dieser Interpretation:
%
\begin{alignat*}{2}
  (\term{VL} \sqcap \term{PflichtVL})^\Imc & = \{v_1,v_2\} \cap \{v_1\}          && = \{v_1\} \\
  (\lnot\term{VL})^\Imc                    & = \Delta^\Imc \setminus \{v_1,v_2\} && = \{s_1,s_2,s_3\} \\
  (\term{Student} \sqcup \term{VL})^\Imc   & = \{s_1,s_2,s_3\} \cup \{v_1,v_2\} && = \Delta^\Imc \\
  (\exists\term{bekanntMit}.\term{Student})^\Imc & = \{s_1,s_2,s_3\} \\
  (\exists\term{bekanntMit}.\exists\term{bekanntMit}.\term{Student})^\Imc & = \{s_1,s_2,s_3\} \\
  (\forall\term{hört}.\term{PflichtVL})^\Imc & = \{s_1,s_3,v_1,v_2\}
\end{alignat*}
%
In der letzten Zeile beachte man die Besonderheit der Werterestriktion ($\forall$),
dass ein Domänenelement $d$, welches \emph{keine} ausgehenden $r$-Kanten besitzt,
immer eine Instanz von $\forall r.C$ ist, für jedes beliebige Konzept $C$.

% ===================================================================
\section*{T2.3~ Semantik von {\boldmath $\top$ und $\bot$}}

Es gelten:
%
\par\vspace*{-2.7\baselineskip}
\begin{alignat*}{4}
  \top^\Imc & = (A \sqcup \lnot A)^\Imc & = A^\Imc \cup (\Delta^\Imc \setminus A^\Imc) & = \Delta^\Imc \\
  \bot^\Imc & = (A \sqcap \lnot A)^\Imc & = A^\Imc \cap (\Delta^\Imc \setminus A^\Imc) & = \emptyset^\Imc
\end{alignat*}
%
Dabei folgt die erste Gleichheit jeder Zeile aus der Definition von $\top$ bzw.\ $\bot$
auf Folie~2.9,
die zweite Gleichheit aus der Semantik (Def.~2.2)
und die dritte aus der Mengenlehre.

\pagebreak
% ===================================================================
\section*{T2.4~ Beispiele für "`unerfüllbar"' und "`subsumiert"'}

\begin{enumerate}
  \item[(a)]
    Das Konzept $C = \exists r.A \sqcap \forall r. \lnot A$ is \emph{nicht} erfüllbar:
    \par\smallskip
    Angenommen, $C$ sei erfüllbar, d.\,h.\ es gibt eine Interpretation $\Imc$
    mit $C^\Imc \neq \emptyset$. Sei $d \in C^\Imc$.
    Wegen $d \in (\exists r.A)^\Imc$ gibt es ein Element $e \in A^\Imc$
    mit $(d,e) \in r^\Imc$. Wegen $d \in (\forall r.\lnot A)^\Imc$
    gilt aber $e \in (\lnot A)^\Imc$, also $e \notin A^\Imc$,
    was ein Widerspruch zu $e \in A^\Imc$ ist. Also ist die Annahme falsch.
    \par\medskip
  \item[(b)]
    $\exists r.(A \sqcap B) \sqsubseteq \exists r.A \sqcap \exists r.B$:
    \par\smallskip
    Sei \Imc eine Interpretation und $d \in (\exists r.(A \sqcap B))^\Imc$.
    Dann gibt es ein Element $e \in (A \sqcap B)^\Imc$ mit $(d,e) \in r^\Imc$.
    Wegen $e \in A^\Imc$ gilt $d \in (\exists r.A)^\Imc$;
    wegen $e \in B^\Imc$ gilt $d \in (\exists r.B)^\Imc$.
    Also ist $d \in (\exists r.A \sqcap \exists r.B)^\Imc$.
    \par\smallskip
    Die Rückrichtung dieser Subsumtion gilt nicht -- finde ein Gegenbeispiel, d.\,h.\
    eine Interpretation $\Imc$ mit
    $(\exists r.A \sqcap \exists r.B)^\Imc \nsubseteq (\exists r.(A \sqcap B))^\Imc$.
\end{enumerate}

% ===================================================================
\section*{T2.5~ Beispiele für TBoxen und deren Semantik}

Wir betrachten folgende TBox.
\[
  \begin{tboxarray}
    \Tmc = \{
      & \term{Student}   & \equiv & \term{Mensch} \sqcap \exists\term{hört}.\term{Vorlesung} & \\
      & \term{Vorlesung} & \equiv & \term{PflichtVL} \sqcup \term{WahlVL}                    & \\
      & \term{Student} \sqcap \exists\term{hört}.\term{Vorlesung} & \sqsubseteq & \exists\term{bekanntMit}.\term{Student} & \\
      & \term{PflichtVL} \sqcap \term{WahlVL} & \sqsubseteq & \bot & \}
  \end{tboxarray}
\]
Die Interpretation aus T2.2 ist Modell von \Tmc.
Sie erfüllt z.\,B.\ auch die folgende Konzeptinklusion.
\begin{equation}
  \label{eq:zusaetzl_KI}
  \term{Student} \sqsubseteq \exists\term{bekanntMit}.\term{Mensch}
\end{equation}

Ein weiteres Modell ist z.\,B. folgende Interpretation \Jmc.
%
\begin{center}
  \begin{tikzpicture}[%
    >=Latex,
    every state/.style={draw=black,thin,fill=black!10,inner sep=1mm,minimum size=8mm},
    every edge/.style={draw=black,thin}
  ]
    \node[state] (s1)                    {$s_1$};
    \node[state] (s2) [right=35mm of s1] {$s_2$};
    \node[state] (v1) [below=15mm of s1] {$v_1$};
    
    \node [above=-1mm of s1] {\begin{tabular}{@{}c@{}}\term{Student}\\\term{Mensch}\end{tabular}};
    \node [above=.4mm of s2]  {\term{Mensch}};
    \node [below=0mm of v1] {\begin{tabular}{@{}c@{}}\term{Vorlesung}\\\term{PflichtVL}\end{tabular}};
    
    \path[->] (s1) edge[loop left] node[left]  {\term{bekanntMit}} ()
              (s1) edge            node[above] {\term{bekanntMit}} (s2)
              (s1) edge            node[right] {\term{hört}}       (v1)
    ;
  \end{tikzpicture}
\end{center}
%
\Jmc erfüllt ebenfalls die Konzeptinklusion~\eqref{eq:zusaetzl_KI}
sowie z.\,B.\ $\term{VL} \equiv \term{PflichtVL}$.

\goodbreak
% ===================================================================
\section*{T2.6~ Beispiele für "`erfüllbar"' und "`subsumiert"' bzgl.\ TBoxen}

Sei \Tmc die TBox aus dem vorangehenden Beispiel.

\begin{enumerate}
  \item[(a)]
    Das Konzept
    \[
      C = \term{Student} \sqcap \forall\term{hört}.\term{PflichtVL}
    \]
    ist erfüllbar bezüglich \Tmc, denn folgende Interpretation 
    $\Imc'$ ist ein Modell von \Tmc mit $s_1 \in C^{\Imc'}$:
    %
    \begin{center}
      \begin{tikzpicture}[%
        >=Latex,
        every state/.style={draw=black,thin,fill=black!10,inner sep=1mm,minimum size=8mm},
        every edge/.style={draw=black,thin}
      ]
        \node[state] (s1)                    {$s_1$};
        \node[state] (v1) [below=10mm of s1] {$v_1$};
        
        \node [above=-1mm of s1] {\begin{tabular}{@{}c@{}}\term{Student}\\\term{Mensch}\end{tabular}};
        \node [below=0mm of v1] {\begin{tabular}{@{}c@{}}\term{Vorlesung}\\\term{PflichtVL}\end{tabular}};
        
        \path[->] (s1) edge[loop left] node[left]  {\term{bekanntMit}} ()
                  (s1) edge            node[right] {\term{hört}}       (v1)
        ;
      \end{tikzpicture}
    \end{center}
    %
    Ebenso ist die Interpretation \Imc aus T2.2 ein Modell von \Tmc mit $s_1 \in C^\Imc$.
    \parII
  \item[(b)]
    Das Konzept
    \[
      C = \term{Student} \sqcap \forall\term{hört}.\term{PflichtVL} \sqcap \exists\term{hört}.\term{WahlVL}
    \]
    ist \emph{un}erfüllbar bezüglich \Tmc:
    Angenommen, $C$ sei erfüllbar bzgl.\ \Tmc.
    Dann gibt es ein Modell \Imc von \Tmc mit einer Instanz $d \in C^\Imc$.
    Nach der Semantik von "`$\sqcap$"' (Def.~2.2) gelten
    (i) $d \in (\forall\term{hört}.\term{PflichtVL})^\Imc$ und
    (ii) $d \in (\exists\term{hört}.\term{WahlVL})^\Imc$.
    Wegen (ii) gibt es ein Element $e \in \term{WahlVL}^\Imc$ mit $(d,e) \in \term{hört}^\Imc$.
    Wegen (i) ist dann auch $e \in \term{PflichtVL}^\Imc$,
    also $e \in (\term{PflichtVL} \sqcap \term{WahlVL})^\Imc$.
    Weil \Imc jedoch ein Modell von \Tmc ist,
    kann es wegen der Konzeptinklusion $\term{PflichtVL} \sqcap \term{WahlVL} \sqsubseteq \bot$
    aus \Tmc ein solches Element $e$ nicht geben; ein Widerspruch.
    Also ist die Annahme falsch.
    \parII
  \item[(c)]
    Für die Konzepte
    \[
      C = \term{Student}
      \qquad\text{und}\qquad
      D = \exists\term{bekanntMit}.\term{Student}
    \]
    gilt $\Tmc \models C \sqsubseteq D$:
    Sei \Imc ein Modell von \Tmc und $d \in C^\Imc$, d.\,h.\ $d \in \term{Student}^\Imc$.
    Zu zeigen ist $d \in D^\Imc$, d.\,h.\ $d \in (\exists\term{bekanntMit}.\term{Student})^\Imc$.
    
    Wegen der ersten Zeile von~\Tmc gilt
    $d \in (\exists\term{hört}.\term{Vorlesung})^\Imc$,
    also auch $d \in (\term{Student} \sqcap \exists\term{hört}.\term{Vorlesung})^\Imc$.
    Mit Zeile~3 von~\Tmc folgt wie gewünscht $d \in (\exists\term{bekanntMit}.\term{Student})^\Imc$.    
\end{enumerate}
%
Dies ist bereits Schlussfolgern, denn wir haben implizites Wissen aus \Tmc abgeleitet:
%
\begin{enumerate}
  \item[(a)]
    Es \emph{kann} Student*innen geben, die nur Pflichtvorlesungen hören.
  \item[(b)]
    Es kann \emph{keine} Student*innen geben, die nur Pflichtvorlesungen,
    aber mindestens eine Wahlvorlesung hören.
  \item[(c)]
    Jede*r Student*in ist mit mindestens einer/m Student*in bekannt.
\end{enumerate}

% ===================================================================
\section*{T2.7~ Beweis der Monotonie von {\boldmath \ALC} (Lemma~2.7)}

\textsfbf{Lemma 2.7}~
Seien $\Tmc_1$ und $\Tmc_2$ TBoxen mit $\Tmc_1 \subseteq \Tmc_2$\,. Dann gilt:
%
\begin{enumerate}
  \item[(1)]
    Wenn $C$ erfüllbar bezüglich $\Tmc_2$ ist, dann ist $C$ erfüllbar bezüglich $\Tmc_1$.
  \item[(2)]
    Wenn $\Tmc_1 \models C \sqsubseteq D$, dann $\Tmc_2 \models C \sqsubseteq D$.
\end{enumerate}

\par\medskip\noindent
\begin{beweis}
  \begin{enumerate}
    \item[\textsfbf{(1)}]
      Sei $C$ erfüllbar bezüglich $\Tmc_2$.
      Dann gibt es ein Modell \Imc von $\Tmc_2$ mit $C^\Imc \neq \emptyset$.
      Da \Imc Modell von $\Tmc_2$ ist, erfüllt \Imc alle Konzeptinklusionen in $\Tmc_2$,
      also wegen $\Tmc_1 \subseteq \Tmc_2$ auch alle Konzeptinklusionen in $\Tmc_1$,
      und somit ist \Imc auch Modell von $\Tmc_1$.
      Also gibt es ein Modell \Imc von $\Tmc_1$ mit $C^\Imc \neq \emptyset$;
      d.\,h.\ $C$ ist erfüllbar bezüglich $\Tmc_1$.
    \item[\textsfbf{(2)}]
      Wir beweisen die Kontraposition.
      Es gelte $\Tmc_2 \not\models C \sqsubseteq D$.%
      \footnote{%
        Das Zeichen $\not\models$ steht für "`nicht $\models$"',
        also bedeutet $\Tmc \not\models C \sqsubseteq D$,
        dass die Beziehung $\Tmc \models C \sqsubseteq D$ \emph{nicht} gilt
        (d.\,h.\ bezüglich \Tmc wird $C$ \emph{nicht} von $D$ subsumiert).
%        Vergleiche auch $=$ versus $\neq$ oder $\subseteq$ versus $\nsubseteq$.
      }
      Dann gibt es ein Modell \Imc von $\Tmc_2$ mit $C^\Imc \nsubseteq D^\Imc$.
      Wie in (1) ist \Imc auch Modell von $\Tmc_1$,
      also $\Tmc_1 \not\models C \sqsubseteq D$.
      \qedhere
  \end{enumerate}
\end{beweis}
%
\par\noindent
Auf der Folie steht auch: "`Die Umkehrungen von~(1) und~(2) sind im Allgemeinen \emph{nicht} richtig."'
Davon kann man sich mittels einfacher Gegenbeispiele überzeugen:
z.\,B.\ ist mit $\Tmc_1=\emptyset$ und $\Tmc_2=\{A \sqsubseteq B\}$
die Umkehrung von~(2) widerlegt, denn $\Tmc_2 \models A \sqsubseteq B$,
aber $\Tmc_1 \not\models A \sqsubseteq B$.

% ===================================================================
\section*{T2.8~ Beispiel für Subsumtion als Ordnungsrelation}

Wir betrachten folgende TBox.
\[
  \begin{tboxarray}
    \Tmc = \{
      & \term{PC}        & \sqsubseteq & \term{Gerät} \sqcap \exists\term{hatTeil}.\term{CPU} & \\
      & \term{PC}        & \equiv      & \term{Desktop} \sqcup \term{Laptop}                  & \\
      & \term{Desktop}   & \sqsubseteq & \lnot\term{Laptop}                                   & \\
      & \term{MobilerPC} & \equiv      & \term{PC} \sqcap \lnot \term{Desktop}                & \}
  \end{tboxarray}
\]
Die dritte Zeile von \Tmc ist äquivalent zu $\term{Desktop} \sqcap \term{Laptop} \sqsubseteq \bot$, wie man leicht zeigt (probiere es selbst aus).

Die Ordnung "`$\sqsubseteq$ bezüglich \Tmc{}"' kann man durch folgendes Hasse-Diagramm darstellen:
%
\begin{center}
  \begin{tikzpicture}[%
    sibling distance=20mm, level distance=10mm,
    every node/.style = {draw=none, fill=none, inner sep=1mm, minimum size=1mm},
    edge from parent/.style = {draw=black, thin, -}%
  ]
    \node(root) {\term{Gerät}}
      child {
        node {\term{PC}}
        child {
          node {\term{Desktop}}
        }
        child {
          node (Laptop) {\term{Laptop}}
        }
      };
      
    \node[below=0mm of Laptop] {\term{MobilerPC}};

    \node[right=30mm of root] {\term{CPU}};    
  \end{tikzpicture}%
\end{center}
%
Dass die Relation "`$\sqsubseteq$ bezüglich \Tmc{}"' \emph{nicht} antisymmetrisch ist,
zeigt sich in diesem Beispiel dadurch, dass der Knoten unten rechts zwei Beschriftungen hat,
also $\Tmc \models \term{Laptop} \equiv \term{MobilerPC}$.
Wäre die Relation antisymmetrisch, dann dürfte nicht gleichzeitig
$\Tmc \models \term{Laptop} \sqsubseteq \term{MobilerPC}$
und $\Tmc \models \term{MobilerPC} \sqsubseteq \term{Laptop}$ gelten.

% ===================================================================
\section*{T2.9~ Beweis der wechselseitigen Reduktionen der Schlussfolgerungsprobleme}

\textsfbf{Lemma 2.9}~
\begin{enumerate}
  \item[(1)]
    Subsumtion ist polynomiell reduzierbar auf (Un)erfüllbarkeit:
    \par\smallskip
    $\Tmc \models C \sqsubseteq D$ ~~gdw.~~ $C \sqcap \neg D$ unerfüllbar bezüglich \Tmc
    \par\smallskip
  \item[(2)]
    Erfüllbarkeit ist polynomiell reduzierbar auf (Nicht-)Äquivalenz:
    \par\smallskip
    $C$ erfüllbar bezüglich \Tmc ~~gdw.~~ $\Tmc \not\models C \equiv \bot$
    \par\smallskip
  \item[(3)]
    Äquivalenz ist polynomiell reduzierbar auf Subsumtion:
    \par\smallskip
    $\Tmc \models C \equiv D$ ~~gdw.~~ $\Tmc \models \top \sqsubseteq (C \sqcap D) \sqcup (\neg C \sqcap \neg D)$
\end{enumerate}

\par\medskip\noindent
\begin{beweis}
  Wir beweisen exemplarisch Punkt~(1). Die Beweise der anderen zwei Punkte sind analog.
  %
%  \begin{center}
%    \renewcommand{\arraystretch}{1.2}
%    \begin{tabular}{@{}lcl@{}}
%      $\Tmc \models C \sqsubseteq D$
%      & gdw. & für alle Modelle \Imc von \Tmc gilt $C^\Imc \subseteq D^\Imc$ \\
%      & gdw. & es gibt kein Modell \Imc von \Tmc mit $C^\Imc \nsubseteq D^\Imc$ \\
%      & gdw. & es gibt kein Modell \Imc von \Tmc mit $C^\Imc \cap (\Delta^\Imc \setminus D^\Imc) \neq \emptyset$ \\
%      & gdw. & es gibt kein Modell \Imc von \Tmc mit $(C \sqcap \lnot D)^\Imc \neq \emptyset$ \\
%      & gdw. & $C \sqcap \lnot D$ unerfüllbar bezüglich \Tmc
%    \end{tabular}
%  \end{center}
%  %
%  Hier gilt das erste "`gdw."' wegen der Definition von Subsumtion (Def.~2.6),
%  und die zweite Zeile ist eine logisch äquivalente Umformulierung der ersten.
%  Die dritte Zeile ist äquivalent zur zweiten, weil für beliebige Mengen $M_1,M_2$ gilt,
%  dass $M_1 \nsubseteq M_2$ gdw.\ $M_1 \cap \overline{M_2} \neq \emptyset$,
%  wobei $\overline{M_2}$ das Komplement von $M_2$ ist.
%  Man überzeuge sich davon anhand eines Venn-Diagramms.
%  Die vierte Zeile ist schließlich äquivalent zur dritten wegen der
%  Semantik von $\sqcap$ und $\lnot$ (Def.~2.2),
%  und die fünfte ist äquivalent dazu wegen der Definition von Unerfüllbarkeit (Def.~2.6).
  \begin{center}
    \renewcommand{\arraystretch}{1.2}
    \begin{tabular}{@{}lcl@{}}
      $\Tmc \models C \sqsubseteq D$
      & gdw. & für alle Modelle \Imc von \Tmc gilt $C^\Imc \subseteq D^\Imc$ \\
      & gdw. & für alle Modelle \Imc von \Tmc gilt  $C^\Imc \cap (\Delta^\Imc \setminus D^\Imc) = \emptyset$ \\
      & gdw. & für alle Modelle \Imc von \Tmc gilt  $(C \sqcap \lnot D)^\Imc = \emptyset$ \\
      & gdw. & $C \sqcap \lnot D$ unerfüllbar bezüglich \Tmc
    \end{tabular}
  \end{center}
  %
  Hier gilt das erste "`gdw."' wegen der Definition von Subsumtion (Def.~2.6),
  und die zweite Zeile ist äquivalent zur ersten, weil für beliebige Mengen $M_1,M_2$ gilt,
  dass $M_1 \subseteq M_2$ gdw.\ $M_1 \cap \overline{M_2} = \emptyset$,
  wobei $\overline{M_2}$ das Komplement von $M_2$ ist.
  Man überzeuge sich davon anhand eines Venn-Diagramms.
  Die dritte Zeile ist schließlich äquivalent zur dritten wegen der
  Semantik von $\sqcap$ und $\lnot$ (Def.~2.2),
  und die vierte ist äquivalent dazu wegen der Definition von Unerfüllbarkeit (Def.~2.6).
  \qedhere
\end{beweis}


% ===================================================================
% ===================================================================
% ===================================================================
\part{Ausdrucksstärke und Modellkonstruktionen}

% ===================================================================
% ===================================================================
% ===================================================================
\part{Tableau-Algorithmen}

% ===================================================================
% ===================================================================
% ===================================================================
\part{Komplexität}

% ===================================================================
% ===================================================================
% ===================================================================
\part{Effiziente Beschreibungslogiken}

% ===================================================================
% ===================================================================
% ===================================================================
\part{ABoxen und Anfragebeantwortung}


%% ===================================================================
%% ===================================================================
%% ===================================================================
%\part*{Anhang}
%\addcontentsline{toc}{part}{Anhang}
%

%\pagebreak
%\addcontentsline{toc}{part}{Literaturverzeichnis}
%\bibliographystyle{babalpha}
%\bibliography{biblio.bib}
\end{document}
