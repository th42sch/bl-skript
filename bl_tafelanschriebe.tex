% !TeX spellcheck = de_DE_frami
\documentclass[fontsize=11pt, twoside=false, numbers=autoenddot]{scrbook}
\usepackage{bl_tafelanschriebe}

\pagestyle{plain}
% \pagestyle{scrheadings}
% % \chead{\headmark}
% \chead{\partmark}
% \renewcommand{\sectionmark}[1]{\markright{\textsl{#1}}}
% \renewcommand{\chaptermark}[1]{\markright{\textsl{#1}}{}}
% \renewcommand{\partmark}[1]{\markright{\textsl{#1}}}
\parindent0pt
\parskip\smallskipamount

\title{Tafelmitschriften zur Vorlesung \glqq Beschreibungslogik\grqq\\ im Sommersemester 2019}
\author{%
  Prof.\ Dr.\ Thomas Schneider\\[1pt]
  AG Theorie der Künstlichen Intelligenz \\[1pt]
  Fachbereich 3 \\
  \includegraphics[width=.4\linewidth]{logo_ub.jpg} \\[\baselineskip]~%
}
\date{Stand: \today}
\publishers{{\large Dieses Dokument ist noch unvollständig und wird regelmäßig aktualisiert.}}

\begin{document}

\maketitle
\tableofcontents

\refstepcounter{part}
% ===================================================================
% ===================================================================
% ===================================================================
\part{Grundlagen}

% ===================================================================
\section*{T2.1~ Beispiele für {\boldmath \ALC}-Konzepte}

Mit den Konzeptnamen
%
\parI
\begin{quote}
  \term{Student}, \term{Naturwissenschaft}, \term{Professor}, \term{Emeritus},
  \term{PflichtVL}, \term{VL}, \term{Einfach}, \mbox{\term{Interessant}},\\
  $A,B$
\end{quote}
%
\parI
und den Rollennamen
%
\parI
\begin{quote}
  \term{studiert}, \term{hält}, \term{hatÜbungsaufgabe},\\
  $r$
\end{quote} 
%
\parI
kann man z.\,B.\ folgende zusammengesetzte \ALC-Konzepte bilden:
%
\parI
\begin{itemize}
  \item
    $\term{Student} \sqcap \exists\term{studiert}.\term{Naturwissenschaft}$ \\
    (beschreibt Studierende, die mindestens eine Naturwissenschaft studieren)
    \parI
  \item
    $\term{Professor} \sqcap \term{Emeritus} \sqcap \forall\term{hält}.\lnot\term{PflichtVL}$ \\
    (beschreibt Professor*innen im Ruhestand, die keine Pflichtvorlesungen halten)
    \parI
  \item 
    $\term{VL} \sqcap \lnot\term{PflichtVL} \sqcap \forall\term{hatÜbungsaufgabe}.(\term{Einfach} \sqcup \term{Interessant})$ \\
    (beschreibt Wahlvorlesungen, bei denen alle Übungsaufgaben einfach oder interessant sind)
    \parI
  \item
    $A \sqcap \exists r.(\lnot B \sqcup \forall r.A)$
\end{itemize}
%
\parI
(Die Beschreibungen in Klammern werden eigentlich erst richtig klar, wenn die Semantik definiert ist.)

% ===================================================================
\section*{T2.2~ Beispiele für Interpretationen und Extensionen}

Wir betrachten die Interpretation $\Imc = (\Delta^\Imc,\cdot^\Imc)$ mit
%
\begin{align*}
  \Delta^\Imc            & = \{s_1,s_2,s_3,v_1,v_2\} \\[4pt]
  \term{Mensch}^\Imc     & = \{s_1,s_2,s_3\}         \\
  \term{Student}^\Imc    & = \{s_1,s_2,s_3\}         \\
  \term{Vorlesung}^\Imc  & = \{v_1,v_2\}             \\
  \term{PflichtVL}^\Imc  & = \{v_1\}                 \\
  \term{WahlVL}^\Imc     & = \{v_2\}                 \\
  \term{hört}^\Imc       & = \{(s_1,v_1),(s_2,v_1),(s_2,v_2),(s_3,v_1)\} \\
  \term{bekanntMit}^\Imc & = \{(s_1,s_2),(s_2,s_1),(s_1,s_1),(s_2,s_2),(s_3,s_3)\}.
\end{align*}
%
Jede Interpretation kann in offensichtlicher Weise als (knoten- und kantenbeschrifteter)
gerichteter Graph aufgefasst werden; für unsere Beispielinterpretation \Imc:
%
\begin{center}
  \begin{tikzpicture}[%
    >=Latex,
    every state/.style={draw=black,thin,fill=black!10,inner sep=1mm,minimum size=8mm},
    every edge/.style={draw=black,thin}
  ]
    \node[state] (s1)                                   {$s_1$};
    \node[state] (s2) [right=35mm of s1]                {$s_2$};
    \node[state] (s3) [right=35mm of s2]                {$s_3$};
    \node[state] (v1) [below right=25mm and 17mm of s1] {$v_1$};
    \node[state] (v2) [right=35mm of v1]                {$v_2$};
    
    \node [above=-1mm of s1] {\begin{tabular}{@{}c@{}}\term{Student}\\\term{Mensch}\end{tabular}};
    \node [above=-1mm of s2] {\begin{tabular}{@{}c@{}}\term{Student}\\\term{Mensch}\end{tabular}};
    \node [above=-1mm of s3] {\begin{tabular}{@{}c@{}}\term{Student}\\\term{Mensch}\end{tabular}};
    \node [below=0mm of v1] {\begin{tabular}{@{}c@{}}\term{Vorlesung}\\\term{PflichtVL}\end{tabular}};
    \node [below=0mm of v2] {\begin{tabular}{@{}c@{}}\term{Vorlesung}\\\term{WahlVL}\end{tabular}};
    
    \path[->] (s1) edge[loop left]     node[left]             {\term{bekanntMit}} ()
              (s1) edge[bend left =10] node[above]            {\term{bekanntMit}} (s2)
              (s2) edge[bend left =10] node[below]            {\term{bekanntMit}} (s1)
              (s2) edge[loop right]    node[right]            {\term{bekanntMit}} ()
              (s3) edge[loop right]    node[right]            {\term{bekanntMit}} ()
              (s1) edge                node[left =1mm]        {\term{hört}}       (v1)
              (s2) edge                node[right=1mm]        {\term{hört}}       (v1)
              (s2) edge                node[right=1mm,pos=.7] {\term{hört}}       (v2)
              (s3) edge                node[right=3mm,pos=.3] {\term{hört}}       (v1)
    ;
  \end{tikzpicture}
\end{center}
%
Beispiele für die Extensionen einiger zusammengesetzter Konzepte in dieser Interpretation:
%
\begin{alignat*}{2}
  (\term{VL} \sqcap \term{PflichtVL})^\Imc & = \{v_1,v_2\} \cap \{v_1\}          && = \{v_1\} \\
  (\lnot\term{VL})^\Imc                    & = \Delta^\Imc \setminus \{v_1,v_2\} && = \{s_1,s_2,s_3\} \\
  (\term{Student} \sqcup \term{VL})^\Imc   & = \{s_1,s_2,s_3\} \cup \{v_1,v_2\} && = \Delta^\Imc \\
  (\exists\term{bekanntMit}.\term{Student})^\Imc & = \{s_1,s_2,s_3\} \\
  (\exists\term{bekanntMit}.\exists\term{bekanntMit}.\term{Student})^\Imc & = \{s_1,s_2,s_3\} \\
  (\forall\term{hört}.\term{PflichtVL})^\Imc & = \{s_1,s_3,v_1,v_2\}
\end{alignat*}
%
In der letzten Zeile beachte man die Besonderheit der Werterestriktion ($\forall$),
dass ein Domänenelement $d$, welches \emph{keine} ausgehenden $r$-Kanten besitzt,
immer eine Instanz von $\forall r.C$ ist, für jedes beliebige Konzept $C$.

% ===================================================================
\section*{T2.3~ Semantik von {\boldmath $\top$ und $\bot$}}

Es gelten:
%
\par\vspace*{-2.7\baselineskip}
\begin{alignat*}{4}
  \top^\Imc & = (A \sqcup \lnot A)^\Imc & = A^\Imc \cup (\Delta^\Imc \setminus A^\Imc) & = \Delta^\Imc \\
  \bot^\Imc & = (A \sqcap \lnot A)^\Imc & = A^\Imc \cap (\Delta^\Imc \setminus A^\Imc) & = \emptyset^\Imc
\end{alignat*}
%
Dabei folgt die erste Gleichheit jeder Zeile aus der Definition von $\top$ bzw.\ $\bot$
auf Folie~2.9,
die zweite Gleichheit aus der Semantik (Def.~2.2)
und die dritte aus der Mengenlehre.

\pagebreak
% ===================================================================
\section*{T2.4~ Beispiele für "`unerfüllbar"' und "`subsumiert"'}

\begin{enumerate}
  \item[(a)]
    Das Konzept $C = \exists r.A \sqcap \forall r. \lnot A$ is \emph{nicht} erfüllbar:
    \par\smallskip
    Angenommen, $C$ sei erfüllbar, d.\,h.\ es gibt eine Interpretation $\Imc$
    mit $C^\Imc \neq \emptyset$. Sei $d \in C^\Imc$.
    Wegen $d \in (\exists r.A)^\Imc$ gibt es ein Element $e \in A^\Imc$
    mit $(d,e) \in r^\Imc$. Wegen $d \in (\forall r.\lnot A)^\Imc$
    gilt aber $e \in (\lnot A)^\Imc$, also $e \notin A^\Imc$,
    was ein Widerspruch zu $e \in A^\Imc$ ist. Also ist die Annahme falsch.
    \par\medskip
  \item[(b)]
    $\exists r.(A \sqcap B) \sqsubseteq \exists r.A \sqcap \exists r.B$:
    \par\smallskip
    Sei \Imc eine Interpretation und $d \in (\exists r.(A \sqcap B))^\Imc$.
    Dann gibt es ein Element $e \in (A \sqcap B)^\Imc$ mit $(d,e) \in r^\Imc$.
    Wegen $e \in A^\Imc$ gilt $d \in (\exists r.A)^\Imc$;
    wegen $e \in B^\Imc$ gilt $d \in (\exists r.B)^\Imc$.
    Also ist $d \in (\exists r.A \sqcap \exists r.B)^\Imc$.
    \par\smallskip
    Die Rückrichtung dieser Subsumtion gilt nicht -- finde ein Gegenbeispiel, d.\,h.\
    eine Interpretation $\Imc$ mit
    $(\exists r.A \sqcap \exists r.B)^\Imc \nsubseteq (\exists r.(A \sqcap B))^\Imc$.
\end{enumerate}

% ===================================================================
\section*{T2.5~ Beispiele für TBoxen und deren Semantik}

Wir betrachten folgende TBox.
\[
  \begin{tboxarray}
    \Tmc = \{
      & \term{Student}   & \equiv & \term{Mensch} \sqcap \exists\term{hört}.\term{Vorlesung} & \\
      & \term{Vorlesung} & \equiv & \term{PflichtVL} \sqcup \term{WahlVL}                    & \\
      & \term{Student} \sqcap \exists\term{hört}.\term{Vorlesung} & \sqsubseteq & \exists\term{bekanntMit}.\term{Student} & \\
      & \term{PflichtVL} \sqcap \term{WahlVL} & \sqsubseteq & \bot & \}
  \end{tboxarray}
\]
Die Interpretation aus T2.2 ist Modell von \Tmc.
Sie erfüllt z.\,B.\ auch die folgende Konzeptinklusion.
\begin{equation}
  \label{eq:zusaetzl_KI}
  \term{Student} \sqsubseteq \exists\term{bekanntMit}.\term{Mensch}
\end{equation}

Ein weiteres Modell ist z.\,B. folgende Interpretation \Jmc.
%
\begin{center}
  \begin{tikzpicture}[%
    >=Latex,
    every state/.style={draw=black,thin,fill=black!10,inner sep=1mm,minimum size=8mm},
    every edge/.style={draw=black,thin}
  ]
    \node[state] (s1)                    {$s_1$};
    \node[state] (s2) [right=35mm of s1] {$s_2$};
    \node[state] (v1) [below=15mm of s1] {$v_1$};
    
    \node [above=-1mm of s1] {\begin{tabular}{@{}c@{}}\term{Student}\\\term{Mensch}\end{tabular}};
    \node [above=.4mm of s2]  {\term{Mensch}};
    \node [below=0mm of v1] {\begin{tabular}{@{}c@{}}\term{Vorlesung}\\\term{PflichtVL}\end{tabular}};
    
    \path[->] (s1) edge[loop left] node[left]  {\term{bekanntMit}} ()
              (s1) edge            node[above] {\term{bekanntMit}} (s2)
              (s1) edge            node[right] {\term{hört}}       (v1)
    ;
  \end{tikzpicture}
\end{center}
%
\Jmc erfüllt ebenfalls die Konzeptinklusion~\eqref{eq:zusaetzl_KI}
sowie z.\,B.\ $\term{VL} \equiv \term{PflichtVL}$.

\goodbreak
% ===================================================================
\section*{T2.6~ Beispiele für "`erfüllbar"' und "`subsumiert"' bzgl.\ TBoxen}

Sei \Tmc die TBox aus dem vorangehenden Beispiel.

\begin{enumerate}
  \item[(a)]
    Das Konzept
    \[
      C = \term{Student} \sqcap \forall\term{hört}.\term{PflichtVL}
    \]
    ist erfüllbar bezüglich \Tmc, denn folgende Interpretation 
    $\Imc'$ ist ein Modell von \Tmc mit $s_1 \in C^{\Imc'}$:
    %
    \begin{center}
      \begin{tikzpicture}[%
        >=Latex,
        every state/.style={draw=black,thin,fill=black!10,inner sep=1mm,minimum size=8mm},
        every edge/.style={draw=black,thin}
      ]
        \node[state] (s1)                    {$s_1$};
        \node[state] (v1) [below=10mm of s1] {$v_1$};
        
        \node [above=-1mm of s1] {\begin{tabular}{@{}c@{}}\term{Student}\\\term{Mensch}\end{tabular}};
        \node [below=0mm of v1] {\begin{tabular}{@{}c@{}}\term{Vorlesung}\\\term{PflichtVL}\end{tabular}};
        
        \path[->] (s1) edge[loop left] node[left]  {\term{bekanntMit}} ()
                  (s1) edge            node[right] {\term{hört}}       (v1)
        ;
      \end{tikzpicture}
    \end{center}
    %
    Ebenso ist die Interpretation \Imc aus T2.2 ein Modell von \Tmc mit $s_1 \in C^\Imc$.
    \parII
  \item[(b)]
    Das Konzept
    \[
      C = \term{Student} \sqcap \forall\term{hört}.\term{PflichtVL} \sqcap \exists\term{hört}.\term{WahlVL}
    \]
    ist \emph{un}erfüllbar bezüglich \Tmc:
    Angenommen, $C$ sei erfüllbar bzgl.\ \Tmc.
    Dann gibt es ein Modell \Imc von \Tmc mit einer Instanz $d \in C^\Imc$.
    Nach der Semantik von "`$\sqcap$"' (Def.~2.2) gelten
    (i) $d \in (\forall\term{hört}.\term{PflichtVL})^\Imc$ und
    (ii) $d \in (\exists\term{hört}.\term{WahlVL})^\Imc$.
    Wegen (ii) gibt es ein Element $e \in \term{WahlVL}^\Imc$ mit $(d,e) \in \term{hört}^\Imc$.
    Wegen (i) ist dann auch $e \in \term{PflichtVL}^\Imc$,
    also $e \in (\term{PflichtVL} \sqcap \term{WahlVL})^\Imc$.
    Weil \Imc jedoch ein Modell von \Tmc ist,
    kann es wegen der Konzeptinklusion $\term{PflichtVL} \sqcap \term{WahlVL} \sqsubseteq \bot$
    aus \Tmc ein solches Element $e$ nicht geben; ein Widerspruch.
    Also ist die Annahme falsch.
    \parII
  \item[(c)]
    Für die Konzepte
    \[
      C = \term{Student}
      \qquad\text{und}\qquad
      D = \exists\term{bekanntMit}.\term{Student}
    \]
    gilt $\Tmc \models C \sqsubseteq D$:
    Sei \Imc ein Modell von \Tmc und $d \in C^\Imc$, d.\,h.\ $d \in \term{Student}^\Imc$.
    Zu zeigen ist $d \in D^\Imc$, d.\,h.\ $d \in (\exists\term{bekanntMit}.\term{Student})^\Imc$.
    
    Wegen der ersten Zeile von~\Tmc gilt
    $d \in (\exists\term{hört}.\term{Vorlesung})^\Imc$,
    also auch $d \in (\term{Student} \sqcap \exists\term{hört}.\term{Vorlesung})^\Imc$.
    Mit Zeile~3 von~\Tmc folgt wie gewünscht $d \in (\exists\term{bekanntMit}.\term{Student})^\Imc$.    
\end{enumerate}
%
Dies ist bereits Schlussfolgern, denn wir haben implizites Wissen aus \Tmc abgeleitet:
%
\begin{enumerate}
  \item[(a)]
    Es \emph{kann} Student*innen geben, die nur Pflichtvorlesungen hören.
  \item[(b)]
    Es kann \emph{keine} Student*innen geben, die nur Pflichtvorlesungen,
    aber mindestens eine Wahlvorlesung hören.
  \item[(c)]
    Jede*r Student*in ist mit mindestens einer/m Student*in bekannt.
\end{enumerate}

% ===================================================================
\section*{T2.7~ Beweis der Monotonie von {\boldmath \ALC} (Lemma~2.7)}

\textsfbf{Lemma 2.7}~
Seien $\Tmc_1$ und $\Tmc_2$ TBoxen mit $\Tmc_1 \subseteq \Tmc_2$\,. Dann gilt:
%
\begin{enumerate}
  \item[(1)]
    Wenn $C$ erfüllbar bezüglich $\Tmc_2$ ist, dann ist $C$ erfüllbar bezüglich $\Tmc_1$.
  \item[(2)]
    Wenn $\Tmc_1 \models C \sqsubseteq D$, dann $\Tmc_2 \models C \sqsubseteq D$.
\end{enumerate}

\par\medskip\noindent
\begin{beweis}
  \begin{enumerate}
    \item[\textsfbf{(1)}]
      Sei $C$ erfüllbar bezüglich $\Tmc_2$.
      Dann gibt es ein Modell \Imc von $\Tmc_2$ mit $C^\Imc \neq \emptyset$.
      Da \Imc Modell von $\Tmc_2$ ist, erfüllt \Imc alle Konzeptinklusionen in $\Tmc_2$,
      also wegen $\Tmc_1 \subseteq \Tmc_2$ auch alle Konzeptinklusionen in $\Tmc_1$,
      und somit ist \Imc auch Modell von $\Tmc_1$.
      Also gibt es ein Modell \Imc von $\Tmc_1$ mit $C^\Imc \neq \emptyset$;
      d.\,h.\ $C$ ist erfüllbar bezüglich $\Tmc_1$.
    \item[\textsfbf{(2)}]
      Wir beweisen die Kontraposition.
      Es gelte $\Tmc_2 \not\models C \sqsubseteq D$.%
      \footnote{%
        Das Zeichen $\not\models$ steht für "`nicht $\models$"',
        also bedeutet $\Tmc \not\models C \sqsubseteq D$,
        dass die Beziehung $\Tmc \models C \sqsubseteq D$ \emph{nicht} gilt
        (d.\,h.\ bezüglich \Tmc wird $C$ \emph{nicht} von $D$ subsumiert).
%        Vergleiche auch $=$ versus $\neq$ oder $\subseteq$ versus $\nsubseteq$.
      }
      Dann gibt es ein Modell \Imc von $\Tmc_2$ mit $C^\Imc \nsubseteq D^\Imc$.
      Wie in (1) ist \Imc auch Modell von $\Tmc_1$,
      also $\Tmc_1 \not\models C \sqsubseteq D$.
      \qedhere
  \end{enumerate}
\end{beweis}
%
\par\noindent
Auf der Folie steht auch: "`Die Umkehrungen von~(1) und~(2) sind im Allgemeinen \emph{nicht} richtig."'
Davon kann man sich mittels einfacher Gegenbeispiele überzeugen:
z.\,B.\ ist mit $\Tmc_1=\emptyset$ und $\Tmc_2=\{A \sqsubseteq B\}$
die Umkehrung von~(2) widerlegt, denn $\Tmc_2 \models A \sqsubseteq B$,
aber $\Tmc_1 \not\models A \sqsubseteq B$.

% ===================================================================
\section*{T2.8~ Beispiel für Subsumtion als Ordnungsrelation}

Wir betrachten folgende TBox.
\[
  \begin{tboxarray}
    \Tmc = \{
      & \term{PC}        & \sqsubseteq & \term{Gerät} \sqcap \exists\term{hatTeil}.\term{CPU} & \\
      & \term{PC}        & \equiv      & \term{Desktop} \sqcup \term{Laptop}                  & \\
      & \term{Desktop}   & \sqsubseteq & \lnot\term{Laptop}                                   & \\
      & \term{MobilerPC} & \equiv      & \term{PC} \sqcap \lnot \term{Desktop}                & \}
  \end{tboxarray}
\]
Die dritte Zeile von \Tmc ist äquivalent zu $\term{Desktop} \sqcap \term{Laptop} \sqsubseteq \bot$, wie man leicht zeigt (probiere es selbst aus).

Die Ordnung "`$\sqsubseteq$ bezüglich \Tmc{}"' kann man durch folgendes Hasse-Diagramm darstellen:
%
\begin{center}
  \begin{tikzpicture}[%
    sibling distance=20mm, level distance=10mm,
    every node/.style = {draw=none, fill=none, inner sep=1mm, minimum size=1mm},
    edge from parent/.style = {draw=black, thin, -}%
  ]
    \node(root) {\term{Gerät}}
      child {
        node {\term{PC}}
        child {
          node {\term{Desktop}}
        }
        child {
          node (Laptop) {\term{Laptop}}
        }
      };
      
    \node[below=0mm of Laptop] {\term{MobilerPC}};

    \node[right=30mm of root] {\term{CPU}};    
  \end{tikzpicture}%
\end{center}
%
Dass die Relation "`$\sqsubseteq$ bezüglich \Tmc{}"' \emph{nicht} antisymmetrisch ist,
zeigt sich in diesem Beispiel dadurch, dass der Knoten unten rechts zwei Beschriftungen hat,
also $\Tmc \models \term{Laptop} \equiv \term{MobilerPC}$.
Wäre die Relation antisymmetrisch, dann dürfte nicht gleichzeitig
$\Tmc \models \term{Laptop} \sqsubseteq \term{MobilerPC}$
und $\Tmc \models \term{MobilerPC} \sqsubseteq \term{Laptop}$ gelten.

% ===================================================================
\section*{T2.9~ Wechselseitige Reduktion der Schlussfolgerungsprobleme}

\textsfbf{Lemma 2.9}~
\begin{enumerate}
  \item[(1)]
    Subsumtion ist polynomiell reduzierbar auf (Un)erfüllbarkeit:
    \par\smallskip
    $\Tmc \models C \sqsubseteq D$ ~~gdw.~~ $C \sqcap \neg D$ unerfüllbar bezüglich \Tmc
    \par\smallskip
  \item[(2)]
    Erfüllbarkeit ist polynomiell reduzierbar auf (Nicht-)Äquivalenz:
    \par\smallskip
    $C$ erfüllbar bezüglich \Tmc ~~gdw.~~ $\Tmc \not\models C \equiv \bot$
    \par\smallskip
  \item[(3)]
    Äquivalenz ist polynomiell reduzierbar auf Subsumtion:
    \par\smallskip
    $\Tmc \models C \equiv D$ ~~gdw.~~ $\Tmc \models \top \sqsubseteq (C \sqcap D) \sqcup (\neg C \sqcap \neg D)$
\end{enumerate}

\par\medskip\noindent
\begin{beweis}
  Wir beweisen exemplarisch Punkt~(1). Die Beweise der anderen zwei Punkte sind analog.
  %
%  \begin{center}
%    \renewcommand{\arraystretch}{1.2}
%    \begin{tabular}{@{}lcl@{}}
%      $\Tmc \models C \sqsubseteq D$
%      & gdw. & für alle Modelle \Imc von \Tmc gilt $C^\Imc \subseteq D^\Imc$ \\
%      & gdw. & es gibt kein Modell \Imc von \Tmc mit $C^\Imc \nsubseteq D^\Imc$ \\
%      & gdw. & es gibt kein Modell \Imc von \Tmc mit $C^\Imc \cap (\Delta^\Imc \setminus D^\Imc) \neq \emptyset$ \\
%      & gdw. & es gibt kein Modell \Imc von \Tmc mit $(C \sqcap \lnot D)^\Imc \neq \emptyset$ \\
%      & gdw. & $C \sqcap \lnot D$ unerfüllbar bezüglich \Tmc
%    \end{tabular}
%  \end{center}
%  %
%  Hier gilt das erste "`gdw."' wegen der Definition von Subsumtion (Def.~2.6),
%  und die zweite Zeile ist eine logisch äquivalente Umformulierung der ersten.
%  Die dritte Zeile ist äquivalent zur zweiten, weil für beliebige Mengen $M_1,M_2$ gilt,
%  dass $M_1 \nsubseteq M_2$ gdw.\ $M_1 \cap \overline{M_2} \neq \emptyset$,
%  wobei $\overline{M_2}$ das Komplement von $M_2$ ist.
%  Man überzeuge sich davon anhand eines Venn-Diagramms.
%  Die vierte Zeile ist schließlich äquivalent zur dritten wegen der
%  Semantik von $\sqcap$ und $\lnot$ (Def.~2.2),
%  und die fünfte ist äquivalent dazu wegen der Definition von Unerfüllbarkeit (Def.~2.6).
  \begin{center}
    \renewcommand{\arraystretch}{1.2}
    \begin{tabular}{@{}lcl@{}}
      $\Tmc \models C \sqsubseteq D$
      & gdw. & für alle Modelle \Imc von \Tmc gilt $C^\Imc \subseteq D^\Imc$ \\
      & gdw. & für alle Modelle \Imc von \Tmc gilt  $C^\Imc \cap (\Delta^\Imc \setminus D^\Imc) = \emptyset$ \\
      & gdw. & für alle Modelle \Imc von \Tmc gilt  $(C \sqcap \lnot D)^\Imc = \emptyset$ \\
      & gdw. & $C \sqcap \lnot D$ unerfüllbar bezüglich \Tmc
    \end{tabular}
  \end{center}
  %
  Hier gilt das erste "`gdw."' wegen der Definition von Subsumtion (Def.~2.6),
  und die zweite Zeile ist äquivalent zur ersten, weil für beliebige Mengen $M_1,M_2$ gilt,
  dass $M_1 \subseteq M_2$ gdw.\ $M_1 \cap \overline{M_2} = \emptyset$,
  wobei $\overline{M_2}$ das Komplement von $M_2$ ist.
  Man überzeuge sich davon anhand eines Venn-Diagramms.
  Die dritte Zeile ist schließlich äquivalent zur dritten wegen der
  Semantik von $\sqcap$ und $\lnot$ (Def.~2.2),
  und die vierte ist äquivalent dazu wegen der Definition von Unerfüllbarkeit (Def.~2.6).
  \qedhere
\end{beweis}

\vspace*{-.5\baselineskip}
\enlargethispage*{15mm}
% ===================================================================
\section*{T2.10~ Beispiel für inverse Rollen}

Betrachte folgende \ALC-TBox:
\[
  \begin{tboxarray}
    \Tmc = \{
      & \term{Professor} & \sqsubseteq & \term{Verrückt} \sqcap \exists\term{gibt}.\term{Vorlesung} & \\
      & \term{Vorlesung} & \sqsubseteq & \forall\term{wirdGegebenVon}.\lnot\term{Verrückt}    & \}
  \end{tboxarray}
\]
Intuitiv sollte \term{Professor} unerfüllbar bezüglich \Tmc sein;
dies ist aber nicht der Fall, weil \term{Professor} in folgendem Modell von \Tmc
eine Instanz hat:
%
\begin{center}
  \begin{tikzpicture}[%
    >=Latex,
    every state/.style={draw=black,thin,fill=black!10,inner sep=1mm,minimum size=8mm},
    every edge/.style={draw=black,thin}
  ]
    \node[state] (p)                   {$p$};
    \node[state] (v) [right=30mm of p] {$v$};
    \node[state] (d) [right=30mm of v] {$d$};
    
    \node [below=-.5mm of p] {\begin{tabular}{@{}c@{}}\term{Professor}\\\term{Verrückt}\end{tabular}};
    \node [below=0mm of v] {\term{Vorlesung}};
    \node [below=0mm of d] {\term{Dozent}};
    
    \path[->] (p) edge node[above] {\term{gibt}} (v)
              (v) edge node[above] {\term{wirdGegebenVon}} (d)
    ;
  \end{tikzpicture}
\end{center}
\goodbreak
In \ALCI kann man die zweite Konzeptinklusion in \Tmc durch
\[
  \term{Vorlesung} \,\sqsubseteq\, \forall\term{gibt}^-.\lnot\term{Verrückt}
\]
ersetzen. Dann wird \term{Professor} unerfüllbar bezüglich \Tmc
(aber \Tmc hat immer noch Modelle).

% ===================================================================
\section*{T2.11~ Beispiele für Zahlenrestriktionen}

Definition einer Hand als ein Organ mit genau fünf Fingern:
\[
  \term{Hand} \,\equiv\, \term{Organ} \sqcap \qnrgeq 5 {\term{hatTeil}}{\term{Finger}}
                                      \sqcap \qnrleq 5 {\term{hatTeil}}{\term{Finger}}
\]
%
\parI
Forderung, dass in jedem Semester mindestens zwei Wahlpflichtmodule angeboten werden:
\[
  \term{Semester} \,\sqsubseteq\, \qnrgeq 2 {\term{angeboten}}{\term{Wahlpflichtmodul}}
\]

% ===================================================================
% ===================================================================
% ===================================================================
\part{Ausdrucksstärke und Modellkonstruktionen}

% ===================================================================
\section*{T3.1~ Beispiele für Bisimulationen}

\begin{enumerate}
  \item
    Für die Interpretationen
    %
    \parI
    \begin{center}
      \parbox[t]{.4\linewidth}{%
        \begin{tikzpicture}[%
          >=Latex,baseline=.2pt,
          every state/.style={draw=black,thin,fill=black!10,inner sep=1mm,minimum size=8mm},
          every edge/.style={draw=black,thin}
        ]
          \node[state] (d) {$d$};
          
          \node [left=0mm of d] (dlab) {$A$};
          
          \node [above left=0mm and 5mm of d] {$\Imc_1$};
          
          \path[->] (d) edge [loop right] node[right] {$r$} ()
          ;
        \end{tikzpicture}%
      }%
      \hspace*{.1\linewidth}
      \parbox[t]{.4\linewidth}{%
        \begin{tikzpicture}[%
          >=Latex,baseline=.2pt,
          every state/.style={draw=black,thin,fill=black!10,inner sep=1mm,minimum size=8mm},
          every edge/.style={draw=black,thin}
        ]
          \node[state]                     (x0)                    {$x_0$};
          \node[state]                     (x1) [right=10mm of x0] {$x_1$};
          \node[state]                     (x2) [right=10mm of x1] {$x_2$};
          \node[state,draw=none,fill=none] (x3) [right=10mm of x2] {$\cdots$};
          
          \node [below=0mm of x0]              {$A$};
          \node [below=0mm of x1]              {$A$};
          \node [below=0mm of x2]              {$A$};

          \node [above left=0mm and 1mm of x0] {$\Imc_2$};
          
          \path[->] (x0) edge node[above] {$r$} (x1)
                    (x1) edge node[above] {$r$} (x2)
                    (x2) edge node[above] {$r$} (x3)
          ;
        \end{tikzpicture}%
      }
    \end{center}
    %
    ist $\rho = \Delta^{\Imc_1} \times \Delta^{\Imc_2}$ eine Bisimulation.
    \parIII
  \item
    Für die Interpretationen
    %
    \parI
    \begin{center}
      \parbox[t]{.4\linewidth}{%
        \begin{tikzpicture}[%
          >=Latex,baseline=.2pt,
          every state/.style={draw=black,thin,fill=black!10,inner sep=1mm,minimum size=8mm},
          every edge/.style={draw=black,thin}
        ]
          \node[state] (d)                  {$d$};
          \node[state] (e) [below=6mm of d] {$e$};
          
          \node [left=0mm of d] {$A$};
          \node [left=0mm of e] {$A,B$};
          
          \node [above left=0mm and 12mm of d] {$\Imc_1$};
          
          \path[->] (d) edge node[right] {$r$} (e)
          ;
        \end{tikzpicture}%
      }%
      \hspace*{.1\linewidth}
      \parbox[t]{.4\linewidth}{%
        \begin{tikzpicture}[%
          >=Latex,baseline=.2pt,
          every state/.style={draw=black,thin,fill=black!10,inner sep=1mm,minimum size=8mm},
          every edge/.style={draw=black,thin}
        ]
          \node[state] (x)                                 {$x$};
          \node[state] (y)  [below left =6mm and 5mm of x] {$y$};
          \node[state] (y') [below right=6mm and 5mm of x] {$y'$};
          
          \node [above right=-2.5mm and .5mm of x]  {$A$};
          \node [left =0mm of y]                {$A,B$};
          \node [right=0mm of y']               {$A,B$};

          \node [above left=0mm and 23mm of x] {$\Imc_2$};
          
          \path[->] (x) edge node[pos=.3,left] {$r$} (y)
                    (x) edge node[pos=.3,right] {$r$} (y')
          ;
        \end{tikzpicture}%
      }
    \end{center}
    %
    \parI
    ist $\rho = \{(d,x),\,(e,y),\,(e,y')\}$ eine Bisimulation.
    \parIII
  \item
    Für die Interpretationen
    %
    \parI
    \begin{center}
      \parbox[t]{.4\linewidth}{%
        \begin{tikzpicture}[%
          >=Latex,baseline=.2pt,
          every state/.style={draw=black,thin,fill=black!10,inner sep=1mm,minimum size=8mm},
          every edge/.style={draw=black,thin}
        ]
          \node[state] (d)                   {$d$};
          \node[state] (e) [below=10mm of d] {$e$};
          
          \node[left =0mm of d] {$A,B$};
          \node[left =0mm of e] {$A$};
          
          \node [above left=0mm and 12mm of d] {$\Imc_1$};
          
          \path[->]
            (d) edge [loop right,right]    node {$r$} ()
            (d) edge [left, bend right=15] node {$s$} (e)
            (e) edge [right,bend right=15] node {$r$} (d)
          ;
        \end{tikzpicture}%
      }%
      \hspace*{.1\linewidth}
      \parbox[t]{.4\linewidth}{%
        \begin{tikzpicture}[%
          >=Latex,baseline=.2pt,
          every state/.style={draw=black,thin,fill=black!10,inner sep=1mm,minimum size=8mm},
          every edge/.style={draw=black,thin}
        ]
          \node[state] (x)                                 {$x$};
          \node[state] (y) [below left =10mm and 5mm of x] {$y$};
          \node[state] (z) [below right=10mm and 5mm of x] {$z$};
          
          \node[right=0mm of x] {$A,B$};
          \node[left =0mm of y] {$A,B$};
          \node[right=0mm of z] {$A$};

          \node [above left=0mm and 23mm of x] {$\Imc_2$};
          
          \path[->]
            (x) edge [left, bend right=15] node {$r$} (y)
            (y) edge [right,bend right=15] node {$r$} (x)
            (x) edge [right] node               {$s$} (z)
            (y) edge [below,bend right=15] node {$s$} (z)
            (z) edge [above,bend right=15] node[pos=.35] {$r$} (y)
          ;
        \end{tikzpicture}%
      }
    \end{center}
    %
    \parI
    ist $\rho = \{(d,x),\,(d,y),\,(e,z)\}$ eine Bisimulation.
\end{enumerate}

% ===================================================================
\section*{T3.2~ Beweis des Bisimulationstheorems}

\textsfbf{Theorem 3.2}~
Seien $\Imc_1,\Imc_2$ Interpretationen, $d_1 \in \Delta^{\Imc_1}$ und $d_2 \in \Delta^{\Imc_2}$.

\par\smallskip
Wenn $(\Imc_1,d_1) \sim (\Imc_2,d_2)$,~ dann gilt f\"ur alle \ALC-Konzepte $C$:
\[
  d_1 \in C^{\Imc_1}\qquad\text{gdw.}\qquad d_2 \in C^{\Imc_2}
\]      

\par\noindent
\begin{beweis}
  Sei $\rho$ eine Bisimulation zwischen $\Imc_1$ und $\Imc_2$ mit $d_1\rho d_2$.
  Wir beweisen die Behauptung per Induktion über die Struktur von $C$.
  %
  \begin{description}
    \item[Induktionsanfang.]
      Hier ist $C=A$ für einen Konzeptnamen $A$.
      Nach Bedingung (1) für Bisimulationen (Definition~3.1) gilt wie gewünscht:
      $d_1 \in A^{\Imc_1}$ gdw. $d_2 \in A^{\Imc_2}$
    \item[Induktionsschritt.]
      Wir müssen fünf Fälle gemäß des äußersten Konstruktors
      von $C$ unterscheiden $(\lnot,\sqcap,\sqcup,\exists,\forall)$.
      Wegen der (leicht nachzuweisenden) Äquivalenzen
      %
      \[
        C \sqcup D  \,\equiv\, \lnot(\lnot C \sqcap \lnot D)
        \qquad\text{und}\qquad
        \forall r.C \,\equiv\, \lnot \exists r.\lnot C
      \]
      %
      genügt es, wenn wir uns auf die drei Fälle $\lnot,\sqcap,\exists$ beschränken.
      %
      \begin{description}
        \item[{\boldmath $C=\lnot D$}]
          ~ %Dann gilt:
          %
          \parI
          \begin{center}
            \begin{tabular}{@{}llp{40mm}l@{}}
              $d_1 \in C^{\Imc_1}$ & gdw. & $d_1 \notin D^{\Imc_1}$ & (Semantik "`$\lnot$"') \\
                                   & gdw. & $d_2 \notin D^{\Imc_2}$ & (Induktionsvoraussetzung) \\
                                   & gdw. & $d_2 \in C^{\Imc_2}$    & (Semantik "`$\lnot$"')
            \end{tabular}
          \end{center}
          \parI
        \item[{\boldmath $C=D \sqcap E$}]
          ~ %Dann gilt:
          %
          \parI
          \begin{center}
            \begin{tabular}{@{}llp{40mm}l@{}}
              $d_1 \in C^{\Imc_1}$
              & gdw. & $d_1 \in D^{\Imc_1}$ und $d_1 \in E^{\Imc_1}$ & (Semantik "`$\sqcap$"') \\
              & gdw. & $d_2 \in D^{\Imc_2}$ und $d_1 \in E^{\Imc_2}$ & (Induktionsvoraussetzung)\\
              & gdw. & $d_2 \in C^{\Imc_2}$                          & (Semantik "`$\sqcap$"')
            \end{tabular}
          \end{center}
          \parI
        \item[{\boldmath $C=\exists r.D$}]
          ~\par
          Für die Richtung "`$\Rightarrow$"' argumentieren wir so:
          %
          \parIII
          $d_1 \in C^{\Imc_1}$ \\
          %
          \begin{tabular}{@{}l@{}l@{~\,}l@{~~}l@{}}
            & $\Rightarrow$ & es gibt $e_1 \in \Delta^{\Imc_1}$ mit $(d_1,e_1) \in r^{\Imc_1}$ und $e_1 \in D^{\Imc_1}$ & (Semantik "`$\exists$"') \\
            & $\Rightarrow$ & es gibt $e_2 \in \Delta^{\Imc_2}$ mit $(d_2,e_2) \in r^{\Imc_2}$ und $e_1\rho e_2$ & (Bedingung~(2) Bisim.) \\
            & $\Rightarrow$ & $e_2 \in D^{\Imc_2}$ & (Induktionsvorauss.) \\
            & $\Rightarrow$ & $d_2 \in (\exists r.D)^{\Imc_2}$ & (Semantik "`$\exists$"')
          \end{tabular}
          %
          \parIII
          Das Argument für die Richtung "`$\Leftarrow$"' ist analog,
          unter Verwendung von Bedingung~(3) für Bisimulationen.
          \qedhere
      \end{description}
  \end{description}
\end{beweis}

% ===================================================================
\section*{T3.3~ Nichtausdrückbarkeit konkreter Eigenschaften}

\textsfbf{Theorem 3.4}~
In \ALC sind \emph{nicht ausdr\"uckbar:}
%
\begin{itemize}
  \item
    das \ALCI-Konzept $\exists r^- . \top$
  \item
    die \ALCQ-Konzepte 
    %
    \begin{itemize}
      \item
        $\qnrleq n r {\top}$,~ f\"ur alle $n > 0$
      \item
        $\qnrgeq n r {\top}$,~ f\"ur alle $n > 1$
    \end{itemize}
\end{itemize}

\par\noindent
\begin{beweis}
  Siehe Proposition~3.3 und~3.4 in~\cite{DLintro}.
\end{beweis}

\goodbreak
% ===================================================================
\section*{T3.4~ Beispiel für ein Baummodell}

Sei $C = A \sqcap \exists s.B \sqcap \forall s.\exists r.A$
und $\Tmc = \{\top \sqsubseteq \exists s.A\}$.
Ein Baummodell von $C$ und \Tmc:
%
\begin{center}
  \begin{tikzpicture}[%
    >=Latex,baseline=.2pt,
    every state/.style={draw=black,thin,fill=black!10,inner sep=1mm,minimum size=4mm},
    every edge/.style={draw=black,thin}
  ]
    \node[state] (eps)                                   {};
    \node[state] (0)   [below left =6mm and 14mm of eps] {};
    \node[state] (1)   [below right=6mm and 14mm of eps] {};
    \node[state] (00)  [below left =7mm and 4mm of 0]    {};
    \node[state] (01)  [below right=7mm and 4mm of 0]    {};
    \node[state] (10)  [below left =7mm and 4mm of 1]    {};
    \node[state] (11)  [below right=7mm and 4mm of 1]    {};
    \node[state] (000) [below=8mm of 00]                 {};
    \node[state] (010) [below=8mm of 01]                 {};
    \node[state] (100) [below=8mm of 10]                 {};
    \node[state] (110) [below=8mm of 11]                 {};
    
    \node [above right=-.5mm and -.5mm of eps] {$A$};
    \node [above left =-.5mm and -.5mm of 0]   {$B$};
    \node [above right=-.5mm and -.5mm of 1]   {$A$};
    \node [right=-.5mm of 00]                  {$A$};
    \node [right=-.5mm of 01]                  {$A$};
    \node [right=-.5mm of 10]                  {$A$};
    \node [right=-.5mm of 11]                  {$A$};
    \node [right=-.5mm of 000]                 {$A$};
    \node [right=-.5mm of 010]                 {$A$};
    \node [right=-.5mm of 100]                 {$A$};
    \node [right=-.5mm of 110]                 {$A$};

    \node [inner sep=0mm,minimum size=4mm,below=8mm of 000] (0000) {$\vdots$};
    \node [inner sep=0mm,minimum size=4mm,below=8mm of 010] (0100) {$\vdots$};
    \node [inner sep=0mm,minimum size=4mm,below=8mm of 100] (1000) {$\vdots$};
    \node [inner sep=0mm,minimum size=4mm,below=8mm of 110] (1100) {$\vdots$};

    \path[->]
      (eps) edge node[pos=.25,left =2mm] {$s$} (0)
      (eps) edge node[pos=.25,right=2mm] {$s$} (1)
      (0)   edge node[pos=.3,left]       {$s$} (00)
      (0)   edge node[pos=.3,right]      {$r$} (01)
      (1)   edge node[pos=.3,left]       {$s$} (10)
      (1)   edge node[pos=.3,right]      {$r$} (11)
      (00)  edge node[right]             {$s$} (000)
      (01)  edge node[right]             {$s$} (010)
      (10)  edge node[right]             {$s$} (100)
      (11)  edge node[right]             {$s$} (110)
      (000) edge node[right]             {$s$} (0000)
      (010) edge node[right]             {$s$} (0100)
      (100) edge node[right]             {$s$} (1000)
      (110) edge node[right]             {$s$} (1100)
    ;
  \end{tikzpicture}%
\end{center}

% ===================================================================
\section*{T3.5~ Beispiel für das Unravelling}

Wir betrachten folgende Interpretation \Imc.
%
\begin{center}
  \begin{tikzpicture}[%
    >=Latex,baseline=.2pt,
    every state/.style={draw=black,thin,fill=black!10,inner sep=1mm,minimum size=8mm},
    every edge/.style={draw=black,thin}
  ]
    \node[state] (d)                   {$d$};
    \node[state] (e) [right=20mm of d] {$e$};
    
    \node[above=0mm of d] {$A,B$};
    \node[above=0mm of e] {$B$};
    
    \node [above left=0mm and 6mm of d] {$\Imc$};
    
    \path[->]
      (d) edge [loop below,below]    node {$r$} ()
      (d) edge [above, bend left=15]  node {$r$} (e)
      (e) edge [below,bend left=15]  node {$s$} (d)
      (e) edge [loop below,below]    node {$r$} ()
    ;
  \end{tikzpicture}%
\end{center}
%
Dann gibt es beispielsweise folgende $d$-Pfade:
%
\begin{itemize}
  \item
    $\rho = ddedee$ mit $\textsf{end}(\rho) = e$
  \item
    $\rho' = deeed$ mit $\textsf{end}(\rho') = d$
\end{itemize}
%
\parII
Das Unravelling von \Imc an Stelle $d$ gemäß Definition~3.7 ist
folgende Interpretation \Jmc.
%
\begin{center}
  \begin{tikzpicture}[%
    >=Latex,baseline=.2pt,
    every state/.style={draw=black,thin,fill=black!10,inner sep=.5mm,minimum size=8.5mm},
    every edge/.style={draw=black,thin}
  ]
    \node[state] (eps)                                   {$d$};
    \node[state] (0)   [below left =6mm and 20mm of eps] {$dd$};
    \node[state] (1)   [below right=6mm and 20mm of eps] {$de$};
    \node[state] (00)  [below left =7mm and 6mm of 0]    {$ddd$};
    \node[state] (01)  [below right=7mm and 6mm of 0]    {$dde$};
    \node[state] (10)  [below left =7mm and 6mm of 1]    {$ded$};
    \node[state] (11)  [below right=7mm and 6mm of 1]    {$dee$};
    
    \node [above right=-1mm and -.5mm of eps]  {$A,B$};
    \node [above left =-1mm and -.5mm of 0]    {$A,B$};
    \node [above right=-.5mm and -.5mm of 1]   {$B$};
    \node [right=-.5mm of 00]                  {\rule{0pt}{10pt}$A,B$};
    \node [right=-.5mm of 01]                  {$B$};
    \node [right=-.5mm of 10]                  {\rule{0pt}{10pt}$A,B$};
    \node [right=-.5mm of 11]                  {$B$};

%    \node [inner sep=0mm,minimum size=4mm,below=8mm of 00] (000) {$\vdots$};
%    \node [inner sep=0mm,minimum size=4mm,below=8mm of 01] (010) {$\vdots$};
%    \node [inner sep=0mm,minimum size=4mm,below=8mm of 10] (100) {$\vdots$};
%    \node [inner sep=0mm,minimum size=4mm,below=8mm of 11] (110) {$\vdots$};
%
    \node [inner sep=0mm,minimum size=4mm,below=0mm of 00] {$\vdots$};
    \node [inner sep=0mm,minimum size=4mm,below=0mm of 01] {$\vdots$};
    \node [inner sep=0mm,minimum size=4mm,below=0mm of 10] {$\vdots$};
    \node [inner sep=0mm,minimum size=4mm,below=0mm of 11] {$\vdots$};

    \node [above left=0mm and 36mm of eps] {$\Jmc$};
    
    \path[->]
      (eps) edge node[pos=.25,left =2mm] {$r$} (0)
      (eps) edge node[pos=.25,right=2mm] {$r$} (1)
      (0)   edge node[pos=.3,left]       {$r$} (00)
      (0)   edge node[pos=.3,right]      {$r$} (01)
      (1)   edge node[pos=.3,left]       {$s$} (10)
      (1)   edge node[pos=.3,right]      {$r$} (11)
%      (00)  edge node[right]             {$s$} (000)
%      (01)  edge node[right]             {$s$} (010)
%      (10)  edge node[right]             {$s$} (100)
%      (11)  edge node[right]             {$s$} (110)
    ;
  \end{tikzpicture}%
\end{center}

\goodbreak
% ===================================================================
\section*{T3.6~ Beweis des Unravelling-Lemmas}

\textsfbf{Lemma 3.8}~
Für alle \ALC-Konzepte $C$ und alle $p \in \Delta^\Jmc$ gilt:
\[
  \textsf{end}(p) \in C^\Imc \qquad\text{gdw.}\qquad p \in C^\Jmc
\]

\par\noindent
\begin{beweis}
  Mit dem Bisimulationstheorem (Theorem~3.2) genügt es zu zeigen,
  dass $\textsf{end}(p)$ und $p$ bisimilar sind,
  d.\,h.\ $\big(\Imc,\textsf{end}(p)\big) \sim (\Jmc,p)$.
  Siehe dazu Lemma~3.22 in~\cite{DLintro}.
\end{beweis}

% ===================================================================
\section*{T3.7~ Beweis der Baummodelleigenschaft von {\boldmath \ALC}}

\textsfbf{Theorem 3.6}~
Wenn ein \ALC-Konzept $C$ bezüglich einer \ALC-TBox \Tmc erfüllbar ist,
dann haben $C$ und \Tmc ein gemeinsames Baummodell \Imc.

\par\noindent
\begin{beweis}
  Siehe Theorem~3.24 in~\cite{DLintro}.
\end{beweis}

% ===================================================================
\section*{T3.8~ Gegenbeispiel für Rückrichtung Bisimulationstheorem}

\textsfbf{Behauptung.}~
Es gibt Interpretationen \Imc und \Jmc und $d \in \Imc$, $e \in \Jmc$, so dass
\begin{enumerate}
  \item[(i)]
    $d \in C^\Imc$ gdw.\ $e \in C^\Jmc$ für alle \ALC-Konzepte $C$,
  \item[(ii)]
    aber $(\Imc,d) \not\sim (\Jmc,e)$.
\end{enumerate}

\par\noindent
\begin{beweis}
  Betrachte die folgenden Interpretationen \Imc und \Jmc.
  %
  \begin{center}
    \dots\ Bild folgt \dots
  \end{center}
  %
  Es gilt (ii):
  versucht man, eine Bisimulation $\rho$ mit $d\mathbin{\rho}e$ zu konstruieren,
  so benötigt man wegen $(e,e') \in r^\Imc$ einen $r$-Nachfolger $d'$ von $d$
  mit $d' \mathbin{\rho} e'$.
  Da jeder $r$-Nachfolger von $d$ aber nur endlich viele weitere Nachfolger hat,
  kann man wegen des unendlichen $r$-Pfads unterhalb von $e'$
  irgendwann nicht mehr Bedingung~(3) von Bisimulationen gewährleisten.
  
  \parII
  Außerdem kann man mittels struktureller Induktion zeigen, dass (i) gilt.
  \qedhere
\end{beweis}

% ===================================================================
\section*{T3.9~ Beispiel für Teilkonzepte}

Sei $C = \forall r.\exists r.(A \sqcap B)$.
Dann ist $\textsf{sub}(C) = \{A,\, B,\, A \sqcap B,\, \exists r.(A \sqcap B),\, \forall r.\exists r.(A \sqcap B)\}$.

\parII
Sei $\Tmc = \{A \sqsubseteq \exists r.B,~ \forall r.B \sqsubseteq A\}$.
Dann ist $\textsf{sub}(\Tmc) = \{A,\, B,\, \exists r.B,\, \forall r.B\}$.

% ===================================================================
\section*{T3.10~ Beispiel für Typen und Filtration}

Seien $C = A \sqcap B$ und $\Tmc = \{A \sqsubseteq \exists r.A\}$.
Dann ist $\textsf{sub}(C,\Tmc) = \{A,\, B,\, A \sqcap B,\, \exists r.A\}$.

Wir betrachten die folgende Interpretation \Imc.
%
\begin{center}
  \begin{tikzpicture}[%
    >=Latex,baseline=.2pt,
    every state/.style={draw=black,thin,fill=black!10,inner sep=1mm,minimum size=8mm},
    every edge/.style={draw=black,thin}
  ]
    \node[state] (d)                   {$d$};
    \node[state] (f) [below=7mm of d] {$f$};
    \node[state] (e) [right=20mm of d] {$e$};
    \node[state] (g) [right=20mm of f] {$g$};
    
    \node[above=0mm of d] {$A$};
    \node[above=0mm of e] {$A$};
    \node[below=0mm of f] {$A,B$};
    \node[below=0mm of g] {$A$};
    
    \node [above left=0mm and 6mm of d] {$\Imc$};
    
    \path[->]
      (d) edge [above,bend left=15]  node {$r$} (e)
      (e) edge [below,bend left=15]  node {$r$} (d)
      (d) edge [left]                node {$r$} (f)
      (e) edge [right]               node {$r$} (g)
      (f) edge [below]               node {$r$} (g)
      
      (e) edge [loop right,right]    node {$r$} ()
      (g) edge [loop right,right]    node {$r$} ()
    ;
  \end{tikzpicture}%
\end{center}
%
Dann gilt für $d$ und die vier Teilkonzepte in $\textsf{sub}(C,\Tmc)$:
$d \in A^\Imc$, $d \notin B^\Imc$, $d \notin (A \sqcap B)^\Imc$ und $d \in (\exists r.A)^\Imc$.
Also ist
\[
  t_\Imc(d) = \{A,\, \exists r.A\}.
\]
Analog erhält man:
\[
  t_\Imc(e) = \{A,\, \exists r.A\}
  \qquad
  t_\Imc(f) = \{A,\, B,\, A \sqcap B,\, \exists r.A\}
  \qquad
  t_\Imc(g) = \{A,\, \exists r.A\}
\]
Es gilt also $d \simeq e \simeq g \not\simeq f$, und somit gibt es zwei Äquivalenzklassen bezüglich $\simeq$\,:
\[
  [d] = \{d,e,g\}
  \qquad
  [f] = \{f\}
\]
Die Filtration von \Imc bezüglich $C$ und \Tmc gemäß Definition~3.17 ist dann
folgende Interpretation \Jmc.
%
\begin{center}
  \begin{tikzpicture}[%
    >=Latex,baseline=.2pt,
    every state/.style={draw=black,thin,fill=black!10,inner sep=1mm,minimum size=8.5mm},
    every edge/.style={draw=black,thin}
  ]
    \node[state] (d)                   {$[d]$};
%    \node[state] (f) [below=7mm of d] {$[f]$};
    \node[state] (f) [right=20mm of d] {$[f]$};
    
    \node[above=0mm of d] {$A$};
    \node[above=0mm of f] {$A,B$};
    
    \node [above left=0mm and 12mm of d] {$\Jmc$};
    
    \path[->]
      (d) edge [above,bend left=15]  node {$r$} (f)
      (f) edge [below,bend left=15]  node {$r$} (d)
      
      (d) edge [loop left,left]      node {$r$} ()
    ;
  \end{tikzpicture}%
\end{center}

%\pagebreak[4]
% ===================================================================
\section*{T3.11~ Beweis des Filtrationstheorems}

\textsfbf{Theorem 3.17}~
Sei \Imc ein Modell von $C$ und \Tmc, und sei \Jmc die Filtration von \Imc bezüglich $C$ und \Tmc.
Dann ist auch \Jmc ein Modell von $C$ und \Tmc.

\par\noindent
\begin{beweis}
  Wir verwenden die folgende Hilfsaussage.
  %
  \begin{quote}
    Für alle $d \in \Delta^\Imc$ und $D \in \textsf{sub}(C,\Tmc)$ gilt:
    \qquad
    $d \in D^\Imc$
    \quad gdw.\quad
    $[d] \in D^\Jmc$
  \end{quote}
  %
  Für den Beweis dieser Hilfsaussage siehe Lemma~3.15 in~\cite{DLintro}.
  
  \parII
  Da \Imc ein Modell von $C$ und \Tmc ist,
  gibt es ein Element $d \in C^\Imc$.
  Mit der Hilfsaussage folgt $[d] \in C^\Jmc$; somit ist $\Jmc$ ein Modell von $C$.
  
  Um zu zeigen, dass $\Jmc$ auch ein Modell von \Tmc ist,
  betrachten wir eine beliebige Konzeptinklusion $D \sqsubseteq E \in \Tmc$
  und eine beliebige Instanz $[d] \in D^\Jmc$.
  Mit der Hilfsaussage folgt $d \in D^\Imc$;
  also $d \in E^\Imc$ (da \Tmc Modell von \Imc ist);
  also mit der Hilfsaussage $[d] \in E^\Jmc$.
  \qedhere
\end{beweis}

% ===================================================================
\section*{T3.12~ {\boldmath \ALCQI} hat nicht die endliche Modelleigenschaft}

Betrachte die TBox $\Tmc = \{(1)~ \top \sqsubseteq \exists r.\neg A,~ (2)~\top \sqsubseteq \qnrleq 1 {r^-} {\top}\}$.
Dann hat der Konzeptname $A$ bezüglich \Tmc nur \emph{unendliche Modelle}:

Sei \Imc ein Modell von $A$ und \Tmc und $d_0 \in \Delta^\Imc$.
Wir müssen zeigen, dass $|\Delta^\Imc| = \infty$.


Da \Imc Modell von \Tmc ist, muss es wegen~(1) einen $r$-Nachfolger $d_1$ von $d_0$ geben
mit $d_1 \in (\lnot A)^\Imc$. Da $d_0 \in A^\Imc$, muss $d_1 \neq d_0$ gelten.

Wegen~(1) muss es wiederum einen $r$-Nachfolger $d_2$ von $d_1$ geben
mit $d_2 \in (\lnot A)^\Imc$. Wie im vorigen Fall muss $d_1 \neq d_0$ gelten.
Außerdem muss $d_2 \neq d_1$ gelten, da sonst $d_1$ zwei $r$-Vorgänger hätte ($d_0$ und $d_1$)
und dann~(2) verletzt wäre.

Dieses Argument kann man so fortsetzen und immer wieder die Existenz eines neuen Elements
$d_{i+1}$ folgern, dass verschieden von allen $d_0,\dots,d_i$ sein muss.
Deshalb muss $\Delta^\Imc$ unendlich sein.

% ===================================================================
\section*{T3.13~ Anzahl der Interpretationen der Größe {\boldmath $\leq 2^n$}}

\textsfbf{Behauptung.}~
Sei $n = |C| + |\Tmc|$.
Dann gibt es höchstens $2^{2^{5n}}$ Interpretationen \Imc mit $|\Delta^\Imc| \leq 2^n$.

\par\noindent
\begin{beweis}
  Zunächst betrachten wir die Anzahl aller Interpretationen \Imc mit $|\Delta^\Imc| = 2^n$.
  
  Jedes Element $d \in \Delta^\Imc$ kann in $A^\Imc$ sein oder nicht,
  für jeden der $\leq n$ Konzeptnamen $A$ in $\textsf{sub}(C,\Tmc)$.
  Also gibt es $2^{2^n \cdot n}$ Möglichkeiten für die Extensionen
  der Konzeptnamen in $C$ und \Tmc.

  Jedes Paar von Elementen $(d,e) \in \Delta^\Imc \times \Delta^\Imc$
  kann in $r^\Imc$ sein oder nicht,
  für jeden der $\leq n$ Rollennamen $r$ in $\textsf{sub}(C,\Tmc)$.
  Also gibt es $2^{2^n \cdot 2^n \cdot n}$ Möglichkeiten für die Extensionen
  der Rollenamen in $C$ und \Tmc.
  
  Die Gesamtzahl der möglichen Extensionen aller Konzept- und Rollennamen
  in $C$ und \Tmc ist dann das Produkt dieser beiden Zahlen, also:
  %
  \begin{align*}
    2^{2^n \cdot n} \cdot 2^{2^n \cdot 2^n \cdot n}
    & = 2^{2^n \cdot n + 2^{2n} \cdot n}             \\
    & \leq 2^{2^{2n} + 2^{3n}}                       \\
    & \leq 2^{2 \cdot 2^{3n}}                        \\
    & = 2^{2^{3n+1}}                                 \\
    & \leq 2^{2^{4n}}
  \end{align*}
  %
  Betrachtet man nun die Anzahl der Interpretationen \Imc mit $|\Delta^\Imc| \leq 2^n$,
  dann sind dies maximal $2^n \cdot 2^{2^{4n}} = 2^{2^{4n}+n} \leq 2^{2^{5n}}$.
  \qedhere
\end{beweis}



% ===================================================================
% ===================================================================
% ===================================================================
\part{Tableau-Algorithmen}

% ===================================================================
\section*{T4.1~ Umwandlung in NNF}

\textsfbf{Lemma 4.2.}~
Jedes Konzept kann in Linearzeit in ein \"aquivalentes Konzept in NNF
umgewandelt werden.

\par\noindent
\begin{beweis}
  Dies geschieht durch erschöpfendes Anwenden folgender Regeln.
  %
  \begin{itemize}
    \item
      Auf"|lösen doppelter Negation:~
      ersetze $\lnot\lnot C$ durch $C$
    \item
      de Morgan:~
      ersetze $\lnot(C \sqcap D)$ durch $\lnot C \sqcup \lnot D$;
      ersetze $\lnot(C \sqcup D)$ durch $\lnot C \sqcap \lnot D$
    \item
      Dualität von $\exists$ und $\forall$:~
      ersetze $\lnot \exists r.C$ durch $\forall r.\lnot C$;
      ersetze $\lnot \forall r.C$ durch $\exists r.\lnot C$
  \end{itemize}
  %
  Jede Regelanwendung ist äquivalenzerhaltend und schiebt die entsprechende Negation
  weiter nach innen (oder löst sie auf).
  
  Man kann zeigen, dass linear viele Regelanwendungen ausreichen.
  \qedhere
\end{beweis}%

% ===================================================================
\section*{T4.2~ Beispiel I-Baum}

Sei $C_0 = A \sqcap \forall r.(\lnot A \sqcap \exists r.B)$.

Dann ist $\textsf{sub}(C_0) = \{A,\,B,\,\lnot A,\,\exists r.B,\,\lnot A \sqcap \exists r.B,\,
\forall r.(\lnot A \sqcap \exists r.B), C_0\}$.

Der folgende Baum ist ein I-Baum für $C_0$.
%
\begin{center}
  \begin{tikzpicture}[%
    >=Latex,baseline=.2pt,
    every state/.style={draw=black,thin,fill=black!10,inner sep=.5mm,minimum size=6mm},
    every edge/.style={draw=black,thin}
  ]
    \node[state] (eps)                                    {};
    \node[state] (0)   [below left =11mm and 20mm of eps] {};
    \node[state] (1)   [below      = 9mm          of eps] {};
    \node[state] (2)   [below right=11mm and 24mm of eps] {};
    \node[state] (00)  [below      = 9mm          of 0]   {};
    
    \node [above right=-4mm and .5mm of eps] {$\{A,B,\exists r.B\}$};
    \node [      left =         0mm of 0]   {$\{B,\lnot A\}$};
    \node [      right=         0mm of 1]   {$\{\exists r.B\}$};
    \node [      right=         0mm of 2]   {$\{A,\lnot A\}$};
    \node [      left =         0mm of 00]  {$\{\exists r.B\}$};

    \path[->]
      (eps) edge node[pos=.3,left =2mm] {$r$} (0)
      (eps) edge node[pos=.5,right]      {$s$} (1)
      (eps) edge node[pos=.3,right=2mm] {$s$} (2)
      (0)   edge node[pos=.5,left]       {$s$} (00)
    ;
  \end{tikzpicture}%
\end{center}
%
Beachte, dass die Knotenbeschriftung im Allgemeinen nicht der Semantik genügen muss;
die Tableau-Regeln werden jedoch dafür sorgen, dass sie das (größtenteils) tut.

% ===================================================================
\section*{T4.3~ Beispiel Tableau-Algorithmus}

Sei $C_0 = (\exists r.A \sqcap \exists r.\lnot A) \sqcap (\forall r.A \sqcup \forall r.B)$.
Im Folgenden ist ein möglicher%
\footnote{%
  Da die Reihenfolge der Regelanwendungen nicht festgelegt ist,
  gibt es im Allgemeinen mehrere Läufe.
  Das Ergebnis (Menge der vollständigen I-Bäume) ist jedoch
  nicht von der Reihenfolge abhängig.%
}
Lauf des Tableau-Algorithmus auf $C_0$ angegeben.
Nach den ersten drei Schritten liefert dieser folgendes Zwischenergebnis:
%
\begin{center}
  \begin{tikzpicture}[%
    >=Latex,baseline=.2pt,
    every state/.style={draw=black,thin,fill=black!10,inner sep=.5mm,minimum size=6mm},
    every edge/.style={draw=black,thin}
  ]
    \node[state] (eps)                                    {};
    
    \node [right=1mm of eps] {%
      \begin{tabular}{@{}ll@{}}
        $C_0$                                  & (1)  \\
        $\exists r.A \sqcap \exists r.\lnot A$ & (2a) \\
        $\forall r.A \sqcup \forall r.B$       & (2b) \\
        $\exists r.A, \exists r.\lnot A$       & (3)
      \end{tabular}
    };

  \end{tikzpicture}%
\end{center}
%
Dabei wurden folgende Regeln angewendet.
%
\begin{center}
  \begin{tabular}{@{}ll@{}}
    (1)        & initialer Baum $B_{\textsf{ini}}$  \\
    (2a), (2b) & $\sqcap$-Regel auf (1)             \\
    (3)        & $\sqcap$-Regel auf (2a)
  \end{tabular}
\end{center}
%
Als nächstes wenden wir die $\sqcup$-Regel auf $\forall r.A \sqcup \forall r.B$ (2b) an.
Dadurch erhalten wir zwei I-Bäume $B_1$ und $B_2$:
%
\begin{center}
  \begin{tikzpicture}[%
    >=Latex,baseline=.2pt,
    every state/.style={draw=black,thin,fill=black!10,inner sep=.5mm,minimum size=6mm},
    every edge/.style={draw=black,thin}
  ]
    \node[state] (eps)                                    {};
    
    \node [right=1mm of eps] {%
      \begin{tabular}{@{}ll@{}}
        $C_0$                                  & (1)  \\
        $\exists r.A \sqcap \exists r.\lnot A$ & (2a) \\
        $\forall r.A \sqcup \forall r.B$       & (2b) \\
        $\exists r.A,~ \exists r.\lnot A$      & (3)  \\
        $\forall r.A$                          & (4a)
      \end{tabular}
    };
    
    \node [above left=0mm and 3mm of eps] {$B_1$};

  \end{tikzpicture}%
  \hspace*{.12\linewidth}
  \begin{tikzpicture}[%
    >=Latex,baseline=.2pt,
    every state/.style={draw=black,thin,fill=black!10,inner sep=.5mm,minimum size=6mm},
    every edge/.style={draw=black,thin}
  ]
    \node[state] (eps)                                    {};
    
    \node [right=1mm of eps] {%
      \begin{tabular}{@{}ll@{}}
        $C_0$                                  & (1)  \\
        $\exists r.A \sqcap \exists r.\lnot A$ & (2a) \\
        $\forall r.A \sqcup \forall r.B$       & (2b) \\
        $\exists r.A,~ \exists r.\lnot A$      & (3)  \\
        $\forall r.B$                          & (4b)
      \end{tabular}
    };

    \node [above left=0mm and 3mm of eps] {$B_2$};

  \end{tikzpicture}%
\end{center}
%
In $B_1$ lässt sich die $\exists$-Regel auf die beiden Konzepte in (3) anwenden,
wodurch jeweils ein neuer $r$-Nachfolger des Wurzelknotens erzeugt wird:
%
\begin{center}
  \begin{tikzpicture}[%
    >=Latex,baseline=.2pt,
    every state/.style={draw=black,thin,fill=black!10,inner sep=.5mm,minimum size=6mm},
    every edge/.style={draw=black,thin}
  ]
    \node[state] (eps)                                    {};
    \node[state] (0) [below left =20mm and 15mm of eps]   {};
    \node[state] (1) [below right=20mm and  0mm of eps]   {};
    
    \node [right=1mm of eps] {%
      \begin{tabular}{@{}ll@{}}
        $C_0$                                  & (1)  \\
        $\exists r.A \sqcap \exists r.\lnot A$ & (2a) \\
        $\forall r.A \sqcup \forall r.B$       & (2b) \\
        $\exists r.A,~ \exists r.\lnot A$      & (3)  \\
        $\forall r.A$                          & (4a)
      \end{tabular}
    };

    \node [right=0mm of 0] {%
      \begin{tabular}{@{}l@{~\,}l@{}}
        $A$ & (5)
      \end{tabular}
    };

    \node [right=0mm of 1] {%
      \begin{tabular}{@{}l@{~\,}l@{}}
        $\lnot A$ & (6)
      \end{tabular}
    };

    \path[->]
      (eps) edge node[pos=.4,left =1mm] {$r$} (0)
      (eps) edge node[pos=.5,left]      {$r$} (1)
    ;
  \end{tikzpicture}%
\end{center}
%
Nun lässt sich die $\forall$-Regel auf (4a) und den rechten Nachfolger anwenden
(der linke ist bereits mit $A$ beschriftet), und wir erhalten:
%
\begin{center}
  \begin{tikzpicture}[%
    >=Latex,baseline=.2pt,
    every state/.style={draw=black,thin,fill=black!10,inner sep=.5mm,minimum size=6mm},
    every edge/.style={draw=black,thin}
  ]
    \node[state] (eps)                                    {};
    \node[state] (0) [below left =20mm and 15mm of eps]   {};
    \node[state] (1) [below right=20mm and  0mm of eps]   {};
    
    \node [right=1mm of eps] {%
      \begin{tabular}{@{}ll@{}}
        $C_0$                                  & (1)  \\
        $\exists r.A \sqcap \exists r.\lnot A$ & (2a) \\
        $\forall r.A \sqcup \forall r.B$       & (2b) \\
        $\exists r.A,~ \exists r.\lnot A$      & (3)  \\
        $\forall r.A$                          & (4a)
      \end{tabular}
    };

    \node [right=0mm of 0] {%
      \begin{tabular}{@{}l@{~\,}l@{}}
        $A$ & (5)
      \end{tabular}
    };

    \node (clash) [right=0mm of 1] {%
      \begin{tabular}{@{}l@{~\,}l@{}}
        $\lnot A$ & (6) \\
        $A$       & (7)
      \end{tabular}
    };
    
    \node [right=0mm of clash] {{\Large \lightning}};

    \path[->]
      (eps) edge node[pos=.4,left =1mm] {$r$} (0)
      (eps) edge node[pos=.5,left]      {$r$} (1)
    ;
  \end{tikzpicture}%
\end{center}
%
Dieser I-Baum ist vollständig (denn es ist keine weitere Regel anwendbar),
und er enthält einen offensichtlichen Widerspruch
wegen (6) und (7) -- ab jetzt immer mit dem Symbol~\lightning\ gekennzeichnet.

\parI
Schließlich wenden wir in $B_2$ dieselben Regeln an und erhalten
folgenden I-Baum:
%
\begin{center}
  \begin{tikzpicture}[%
    >=Latex,baseline=.2pt,
    every state/.style={draw=black,thin,fill=black!10,inner sep=.5mm,minimum size=6mm},
    every edge/.style={draw=black,thin}
  ]
    \node[state] (eps)                                    {};
    \node[state] (0) [below left =20mm and 15mm of eps]   {};
    \node[state] (1) [below right=20mm and  0mm of eps]   {};
    
    \node [right=1mm of eps] {%
      \begin{tabular}{@{}ll@{}}
        $C_0$                                  & (1)  \\
        $\exists r.A \sqcap \exists r.\lnot A$ & (2a) \\
        $\forall r.A \sqcup \forall r.B$       & (2b) \\
        $\exists r.A,~ \exists r.\lnot A$      & (3)  \\
        $\forall r.B$                          & (4b)
      \end{tabular}
    };

    \node [right=0mm of 0] {%
      \begin{tabular}{@{}l@{~\,}l@{}}
        $A$ & (8) \\
        $B$ & (10)
      \end{tabular}
    };

    \node [right=0mm of 1] {%
      \begin{tabular}{@{}l@{~\,}l@{}}
        $\lnot A$ & (9) \\
        $B$       & (11)
      \end{tabular}
    };
    
    \path[->]
      (eps) edge node[pos=.4,left =1mm] {$r$} (0)
      (eps) edge node[pos=.5,left]      {$r$} (1)
    ;
  \end{tikzpicture}%
\end{center}
%
Es wurden also folgende Regeln angewendet:
%
\begin{center}
  \begin{tabular}{@{}ll@{}}
    (8)  & $\exists$-Regel auf $\exists r.A$ (3)               \\
    (9)  & $\exists$-Regel auf $\exists r.\lnot A$ (3)         \\
    (10) & $\forall$-Regel auf (4b) und linken $r$-Nachfolger  \\
    (11) & $\forall$-Regel auf (4b) und rechten $r$-Nachfolger
  \end{tabular}
\end{center}
%
Dieser Baum ist vollständig, enthält aber \emph{keinen} offensichtlichen Widerspruch.
Somit gibt der Algorithmus auf der Eingabe $C_0$ "`erfüllbar"' zurück.

% ===================================================================
\section*{T4.4~ Verzweigungsgrad der I-Bäume}

\textsfbf{Behauptung 1.}~
Es werden nur I-Bäume mit einem
Verzweigungsgrad von maximal $|C_0|$ generiert.

\par\noindent
\begin{beweis}
  Nur die $\exists$-Regel generiert Nachfolgerknoten,
  und zwar höchstens einen pro Konzept $\exists r.C$ in $\textsf{sub}(C_0)$.
  Nach Lemma 3.13 ist aber $|\textsf{sub}(C_0)| \leq C_0$.
  \qedhere
\end{beweis}%

% ===================================================================
\section*{T4.5~ Tiefe der I-Bäume}

\textsfbf{Behauptung 2.}~
Es werden nur I-Bäume mit einer
Tiefe von maximal $|C_0|$ generiert.

\par\noindent
\begin{beweis}
  Dazu genügt es, folgende Behauptung zu beweisen:
  
  \parII
  \textsfbf{Behauptung 2a.}
  Wenn $v$ ein Knoten mit Tiefe $i$ ist,
  dann gilt für alle $C \in \Lmc(v)$:
  \[
    \textsf{rd}(C) \leq \textsf{rd}(C_0) - i
    \tag{$(*)$}
  \]
  
  \parII
  Dabei zählen wir die Tiefe eines Knotens beginnend von der Wurzel,
  welche die Tiefe 0 hat. Behauptung~2a besagt also, dass die Rollentiefe
  der Konzepte in den Knotenbeschriftungen mit der Tiefe eines Knotens abnimmt.
  
  Es ist leicht zu sehen, dass die gewünschte Behauptung~2
  bereits aus Behauptung~2a folgt:
  Wegen der Tableau-Regeln ist jeder Knoten mit mindestens einem Konzept
  beschriftet (der initiale Knoten mit $C_0$ und alle weiteren Knoten mit dem
  $C$ aus der $\exists$-Regel).
  Wenn ein generierter I-Baum also eine Tiefe $k > |C_0|$ hätte,
  dann gäbe es einen Knoten $v$ der Tiefe $k$ und in dessen Beschriftung $\Lmc(v)$
  ein Konzept $C$. Wegen Behauptung~2a wäre dann aber $\textsf{rd}(C) < 0$,
  was nicht möglich ist.
  
  \parII
  \textsfbf{Beweis von Behauptung~2a.}~
  Wir verwenden Induktion über die Anzahl der Regelanwendungen.
  %
  \begin{description}
    \item[Induktionsanfang.]
      Nach 0 Regelanwendungen gibt es nur den initialen Knoten $v_{\textsf{ini}}$
      mit $\Lmc(v_{\textsf{ini}}) = \{C_0\}$.
      Ungleichung~$(*)$ folgt, da $i=0$.
    \item[Induktionsschritt.]
      Hier unterscheiden wir vier Fälle nach der Regel, durch deren Anwendung
      das Konzept $C$ zur Knotenbeschriftung $\Lmc(v)$ hinzugefügt wurde.
      %
      \begin{description}
        \item[$\sqcap$-Regel.]
          Vor der Anwendung der Regel gab es ein Konzept $C \sqcap D \in \Lmc(v)$,
          und durch die Anwendung wurden $C$ und $D$ zu $\Lmc(v)$ hinzugefügt.
          Nach Induktions\-voraussetzung ist
          $\textsf{rd}(C \sqcap D) \leq \textsf{rd}(C_0) - i$,
          also auch $\textsf{rd}(C) \leq \textsf{rd}(C_0) - i$
          wegen $\textsf{C} \leq \textsf{rd}(C \sqcap D)$.
          Analog für $\textsf{rd}(D)$.
        \item[$\sqcup$-Regel.]
          Analog (probiert es selbst aus).
        \item[$\exists$-Regel.]
          Vor der Anwendung der Regel gab es einen Vorgängerknoten $v'$ von $v$
          mit $\exists r.C \in \Lmc(v')$,
          und durch die Anwendung wurde $(v',r,v)$ zu $E$ hinzugefügt
          und $\Lmc(v)=\{C\}$ gesetzt.
          Nun gilt:
          %
          \begin{xalignat*}{2}
            \textsf{rd}(C)
            & = \textsf{rd}(\exists r.C) - 1 & & \text{(Definition \textsf{rd})} \\
            & \leq \Big(\textsf{rd}(C_0) - (i-1)\Big) - 1 & & \text{(Induktionsvorauss.; $v'$ hat Tiefe $i-1$)} \\
            & = \textsf{rd}(C_0) - i
          \end{xalignat*}
        \item[$\forall$-Regel.]
          Ähnlich (probiert es selbst aus).
          \qedhere
      \end{description}
  \end{description}
\end{beweis}%

% ===================================================================
\section*{T4.6~ Anzahl der Regelanwendungen pro I-Baum}

Es bleibt nur noch zu zeigen, dass für $k := |C_0|$ gilt:
\[
  k^k \cdot k \leq 2^{2k^2}
\]
Das kann man wie folgt sehen:
%
\begin{xalignat*}{2}
  k^k \cdot k & = k^{k+1}                      & & \text{(Potenzgesetze)} \\
              & \leq k^{2k}                    & & \text{(da $k \leq 1$)} \\
              & = {\big(2^{\textsf{log} k}\big)}^{2k}  & & \text{(Definition Zweierlogarithmus \textsf{log})}    \\
              & = 2^{\textsf{log} k \cdot 2k}  & & \text{(Potenzgesetze)}    \\
              & \leq 2^{k \cdot 2k}            & & \text{(da $\textsf{log}(k) \leq k$)} \\
              & = 2^{2k^2}                     & &
\end{xalignat*}
%
\qedhere

% ===================================================================
\section*{T4.7~ Letzter Schritt im Terminierungsbeweis}

Wir ordnen jeder Menge $M_i$ von I-Bäumen
eine Multimenge $\textsf{MM}_i$ wie folgt zu:
Für jeden Baum $B \in M_i$ enthält $\textsf{MM}_i$ die Zahl
%
\begin{quote}
  $m(B) = {}$%
  "`$n$ minus die Anzahl $j$ der Regelanwendungen,
  mittels derer $B$ generiert wurde"'.
\end{quote}
%
Somit ist $\textsf{MM}_i$ eine Multimenge über der
Grundmenge $\mathbb{N}$.
Da $<$ auf $\mathbb{N}$ wohlfundiert ist,
ist mit Theorem~4.7 auch $<_{\text{mul}}$ auf $\textsf{MM}(\mathbb{N})$
wohlfundiert.
Außerdem gilt $\textsf{MM}_i <_{\textsf{mul}} \textsf{MM}_{i+1}$ für jedes $i \geq 0$,
denn mit jeder Regelanwendung wird in $M_i$ ein I-Baum durch maximal zwei I-Bäume $B_1,B_2$ ersetzt mit $m(B_1),m(B_2) < m(B)$;
somit erhält man $\textsf{MM}_{i+1}$ aus $\textsf{MM}_i$,
indem man $m(B)$ durch die kleineren Zahlen $m(B_1),m(B_2)$ ersetzt.

Wegen der Wohlfundiertheit von $<_{\text{mul}}$ und der Beobachtung
$\textsf{MM}_i <_{\textsf{mul}} \textsf{MM}_{i+1}$
muss die Folge der $\textsf{MM}_i$ endlich sein.
\qedhere

% ===================================================================
% ===================================================================
% ===================================================================
\part{Komplexität}

% ===================================================================
% ===================================================================
% ===================================================================
\part{Effiziente Beschreibungslogiken}

% ===================================================================
% ===================================================================
% ===================================================================
\part{ABoxen und Anfragebeantwortung}


%% ===================================================================
%% ===================================================================
%% ===================================================================
%\part*{Anhang}
%\addcontentsline{toc}{part}{Anhang}
%

\pagebreak
\addcontentsline{toc}{part}{Literaturverzeichnis}
\bibliographystyle{babalpha}
\bibliography{\jobname.bib}
\end{document}
